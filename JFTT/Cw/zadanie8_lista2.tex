\documentclass{article}
\usepackage[utf8]{inputenc}
\usepackage[T1]{fontenc}
\usepackage{lmodern}
\usepackage[polish]{babel}
\usepackage{amsmath}
\usepackage{tikz}
\usepackage{algorithm}
\usepackage{algpseudocode}
\usepackage{hyperref}
\usepackage{float}
\usepackage{graphicx}
\usepackage{mathtools}
\usepackage{amsmath}
\usepackage{amsfonts}
\usepackage{amsmath}
\usepackage{amsmath}
\usepackage{booktabs}
\usepackage[margin=1in]{geometry}




\title{Języki Formalne i Techniki Translacji}
\author{Wojciech Typer 279730}
\date{}

\begin{document}
\maketitle

\section*{Zadanie 8 Lista 2}
\begin{center}
    Udowodnij, że klasa języków regularnych jest zamknięta na operację różnicy (zbiorów)
\end{center}
Chcemy udowodnić, że jeżeli języki $L_1 i L_2$ są regularne, to język $L_1 - L_2$ (rozumiany jako różnica zbiorów) też jest regularny. \\
Podstawą naszego dowodu będzie tożsamość z teorii zbiorów: $L_1 - L_2 = L_1 \cap \overline{L_2}$, gdzie $\overline{L_2}$ oznacza dopełnienie języka $L_2$, 
czyli zbiór wszystkich słów nad alfabetem języka $L_2$, które nie należą do $L_2$. \\
Teraz musimy udowodnić: 
\begin{itemize}
    \item Jeśli język $L_2$ jest regularny, to jego dopełnienie $\overline{L_2}$ też jest regularne. 
    \item Jeśli $L_1 i \overline{L_2}$ są regularne, to ich przecięcie $L_1 \cap \overline{L_2}$ też jest regularne.
\end{itemize}
\textbf{Dowód: Jeśli język $L_2$ jest regularny, to jego dopełnienie $\overline{L_2}$ też jest regularne} \\
Skoro język $L_2$ jest regularny, to istnieje dla niego automat skończony deterministyczny (DFA) $M = (Q, \Sigma, \delta, q_0, F)$, który go rozpoznaje.
Automat M akceptuje słowo $w$ wtedy i tylko wtedy, gdy po przetworzeniu w kończy w jednym ze stanów akceptujących z $F$. \\
Żeby skonstruować automat rozpoznający dopełnienie języka $L_2$, wystarczy zamienić stany akceptujące na nieakceptujące i odwrotnie:
\begin{center}
    $M_2 = (Q, \Sigma, \delta, q_0, \overline{F})$
\end{center}
gdzie $\overline{F} = Q - F$ \\
W ten sposób udało się nam skonstruować automat $M_2$, który akceptuje słowo $w$ wtedy i tylko wtedy, gdy $M$ go nie akceptuje, czyli dokładnie wtedy, gdy $w$ należy do dopełnienia języka $L_2$. \\
Zatem $\overline{L_2}$ jest regularny. \\
\textbf{Dowód: Jeśli $L_1 i \overline{L_2}$ są regularne, to ich przecięcie $L_1 \cap \overline{L_2}$ też jest regularne.} \\
Skoro języki $L_1$ i $\overline{L_2}$ są regularne, to istnieją dla nich automaty skończone deterministyczne. \\
Niech:
\begin{itemize}
    \item $M_1 = (Q_1, \Sigma, \delta_1, q_{0_1}, F_1)$ - automat rozpoznający język $L_1$
    \item $M_2 = (Q_2, \Sigma, \delta_2, q_{0_2}, F_2)$ - automat rozpoznający język $\overline{L_2}$
\end{itemize}
Teraz skonstruujmy automat, który będzie akceptował słowa należące do $L_1 \cap \overline{L_2}$. \\
Nazwijmy ten automat jako $M_p$ i zdefiniujmy go następująco:
\begin{center}
    $M_p = (Q_p, \Sigma, \delta_p, q_{p_0}, F_p)$, gdzie:
\end{center}
\begin{itemize}
    \item $Q_p$ to iloczyn kartezjański stanów z automatów $M_1$ i $M_2$: $Q_p = Q_1 \times Q_2$
    \item Alfabet $\Sigma$ pozostaje ten sam
    \item Funkcja przejścia $\delta_p$ dla każdego stanu $(q_1, q_2)$ i symbolu $a \in \Sigma$, nowe przejście jest zdefiniowane jako: $\delta_p((q_1, q_2), a) = (\delta_1(q_1, a), \delta_2(q_2, a))$
    \item Stan początkowy $q_{p_0}$ to para stanów początkowych z automatów $M_1$ i $M_2$: $q_{p_0} = (q_{0_1}, q_{0_2})$
    \item Zbiór stanów akceptujących $F_p$: Słowo należy do przecięcia wtedy i tylko wtedy, gdy jest akceptowane przez oba automaty. Dlatego stan w automacie $M_p$ jest akceptujący, gdy oba stany w parze są akceptujące: $F_p = F_1 \times F_2 = \{(q_1, q_2) | q_1 \in F_1 \text{ i } q_2 \in F_2\}$ 
\end{itemize}
Skonstruowaliśmy automat $M_p$, co dowodzi, że przecięcie języków regularnych $L_1$ i $\overline{L_2}$ jest również językiem regularnym. \\
\end{document}