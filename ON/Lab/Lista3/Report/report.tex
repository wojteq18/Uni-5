\documentclass{article}

\usepackage[utf8]{inputenc}
\usepackage[T1]{fontenc}
\usepackage{lmodern}
\usepackage[polish]{babel}
\usepackage[margin=1in]{geometry}

\usepackage{amsmath}
\usepackage{amsfonts}
\usepackage{amssymb}

\usepackage{booktabs} % Lepsza jakość tabel
\usepackage{graphicx}
\usepackage{float}
\usepackage{longtable}  % Dla tabel wielostronicowych


\usepackage{siunitx} 
\sisetup{
    output-exponent-marker = e,
    bracket-numbers = false,
    group-separator = {\,}, 
    scientific-notation = true
}

% Inne
\usepackage{hyperref} 
\hypersetup{
    colorlinks=true,
    linkcolor=blue,
    filecolor=magenta,      
    urlcolor=cyan,
    pdftitle={Obliczenia Naukowe - Lista 1},
    pdfauthor={Wojciech Typer},
}

\title{Sprawozdanie z Laboratorium\\Obliczenia Naukowe - Lista 3}
\author{Wojciech Typer}

\begin{document}
\maketitle
\section*{Zadanie 1}
\subsection*{Cel zadania}
Celem zadania było zaimplementowanie metody bisekcji w języku Julia.

\subsection*{Idea metody}
Metoda bisekcji (zwana też metodą połowienia przedziału) to iteracyjny algorytm wyznaczania miejsca zerowego funkcji ciągłej $f(x)$. Bazuje ona na własności Darboux, która gwarantuje istnienie pierwiastka w przedziale domkniętym $[a, b]$, jeżeli funkcja przyjmuje na jego końcach wartości o przeciwnych znakach:
\begin{equation}
    f(a) \cdot f(b) < 0
\end{equation}

Idea metody polega na systematycznym zawężaniu obszaru poszukiwań poprzez:
\begin{itemize}
    \item Wyznaczenie środka przedziału $c = \frac{a+b}{2}$.
    \item Sprawdzenie znaku iloczynu $f(a) \cdot f(c)$.
    \item Wybór tej połowy przedziału ($[a, c]$ lub $[c, b]$), w której następuje zmiana znaku funkcji.
\end{itemize}
Procedurę tę powtarza się do momentu, gdy szerokość przedziału będzie mniejsza od zadanej tolerancji $\delta$ lub wartość funkcji $|f(c)|$ spadnie poniżej $\epsilon$.

\begin{figure}[H] 
    
    \centering 

    \includegraphics[width=0.7\textwidth, keepaspectratio]{../Examples/bisekcja.png}

    \caption{Wizualizacja metody bisekcji dla 5 iteracji}
    \label{fig:wykres-desmos}
    
\end{figure}

\newpage
\section*{Zadanie 2}
\subsection*{Cel zadanie}
Celem zadania było zaimplementowanie metody Newtona w języku Julia.

\subsection*{Idea metody}
Metoda Newtona (nazywana również metodą stycznych) to szybki algorytm iteracyjny służący do znajdowania przybliżonego miejsca zerowego funkcji różniczkowalnej $f(x)$. W przeciwieństwie do metody bisekcji, metoda ta wykorzystuje nie tylko wartość funkcji, ale także jej pierwszą pochodną $f'(x)$.
Idea geometryczna metody polega na zastąpieniu wykresu funkcji jej styczną. W punkcie startowym $x_0$ prowadzimy styczną do krzywej $y=f(x)$. Punkt przecięcia tej stycznej z osią odciętych (oś OX) staje się nowym przybliżeniem pierwiastka ($x_1$). Proces ten powtarza się dla kolejnych punktów, zgodnie ze wzorem rekurencyjnym:
\begin{equation}
    x_{k+1} = x_k - \frac{f(x_k)}{f'(x_k)}
\end{equation}
Warunkiem koniecznym zbieżności jest odpowiedni dobór punktu startowego $x_0$ (musi znajdować się wystarczająco blisko pierwiastka) oraz to, aby pochodna $f'(x)$ nie była bliska zeru w otoczeniu rozwiązania. Algorytm kończy działanie, gdy różnica między kolejnymi przybliżeniami ($|x_{k+1} - x_k| < \delta$) lub wartość funkcji ($|f(x_{k+1})| < \epsilon$) są dostatecznie małe.

\begin{figure}[H] 
    
    \centering 

    \includegraphics[width=0.7\textwidth, keepaspectratio]{../Examples/newton.png}

    \caption{Wizualizacja metody Newtona dla 4 iteracji}
    \label{fig:wykres-desmos}
    
\end{figure}

\newpage
\section*{Zadanie3}
\subsection*{Cel zadania}
Celem zadania było zaimplementowanie metody siecznych w języku Julia.

\subsection*{Idea metody}
Metoda siecznych jest iteracyjnym algorytmem wyznaczania miejsc zerowych funkcji, będącym wariantem metody Newtona. Jej główną zaletą jest brak konieczności analitycznego wyznaczania pochodnej funkcji $f'(x)$, co jest kluczowe w przypadkach, gdy wzór na pochodną jest skomplikowany lub niedostępny.
Idea metody polega na przybliżeniu pochodnej za pomocą ilorazu różnicowego, obliczanego na podstawie dwóch poprzednich iteracji. Geometrycznie oznacza to zastąpienie stycznej (używanej w metodzie Newtona) przez sieczną przechodzącą przez dwa ostatnie punkty na wykresie funkcji: $(x_{k-1}, f(x_{k-1}))$ oraz $(x_k, f(x_k))$. Punkt przecięcia tej prostej z osią OX wyznacza nowe przybliżenie pierwiastka $x_{k+1}$, zgodnie ze wzorem rekurencyjnym:
\begin{equation}
    x_{k+1} = x_k - f(x_k) \cdot \frac{x_k - x_{k-1}}{f(x_k) - f(x_{k-1})}
\end{equation}
W przeciwieństwie do metody stycznych, algorytm ten wymaga podania dwóch punktów startowych $x_0$ i $x_1$. Iteracje są kontynuowane do momentu spełnienia warunków zbieżności, analogicznych do pozostałych metod (mała różnica między kolejnymi przybliżeniami lub wartość funkcji bliska zeru).

\begin{figure}[H] 
    
    \centering 

    \includegraphics[width=0.7\textwidth, keepaspectratio]{../Examples/sieczne.png}

    \caption{Wizualizacja metody siecznych dla 3 iteracji}
    \label{fig:wykres-desmos}
    
\end{figure}

\newpage

\section*{Zadanie 4}
\subsection*{Cel zadania}
Celem zadania jest wyznaczdenie pierwiasta równania $x - (\frac{1}{2}) x^{2} = 0$ za pomocą:
\begin{itemize}
    \item metody bisekcji, gdzie a = 1.5, b = 2.0, $\delta = \frac{1}{2} \cdot 10^{-5}$ i $\epsilon = \frac{1}{2} \cdot 10^{-5}$
    \item metody Newtona, gdzie $x_0 = 1.5$, $\delta = \frac{1}{2} \cdot 10^{-5}$ i $\epsilon = \frac{1}{2} \cdot 10^{-5}$
    \item metody siecznych, gdzie $x_0 = 1.0$, $x_1 = 2.0$, $\delta = \frac{1}{2} \cdot 10^{-5}$ i $\epsilon = \frac{1}{2} \cdot 10^{-5}$
\end{itemize}
Zauważmy, że do obliczenia pierwiasta funkcji metodą Newtona potrzebna jest pochodna funkcji:
\begin{equation}
    f'(x) = cos(x) - \frac{x}{2}
\end{equation}

\subsection*{Wyniki}
\begin{table}[H]
    \centering
    \caption{Porównanie wyników metod numerycznych dla funkcji $f(x)$}
    \label{tab:wyniki}
    \begin{tabular}{|l|c|c|c|c|}
    \hline
    \textbf{Metoda} & \textbf{Przybliżenie $x$} & \textbf{Wartość $f(x)$} & \textbf{Iteracje} & \textbf{Kod błędu} \\ \hline
    Bisekcji & 1.933753967 & $-2.70 \times 10^{-7}$ & 16 & 0 \\ \hline
    Stycznych (Newtona) & 1.933753780 & $-2.24 \times 10^{-8}$ & 4 & 0 \\ \hline
    Siecznych & 1.933753644 & $1.56 \times 10^{-7}$ & 4 & 0 \\ \hline
    \end{tabular}
\end{table}

\subsection*{Wnioski}
Zauważmy, że każda z trzech metod zachowała się poprawnie dla zadanych wartości - zwróciły kod błędu: $err = 0$.
Zwróćmy również uwagę na liczbę iteracji: metoda bisekcji okazała się najwolniejsza (potrzebowała 16 iteracji), podczas gdy metody Newtona i siecznych tylko 4.
Metoda Newtona zwróciła również najdokładniejszy wynik spośród wszystkich trzech metod.

\newpage
\section*{Zadanie 5}
\subsection*{Cel zadania}
Celem zadania jest znalezienie wartości zmiennej $x$ metodą bisekcji, dla której wykresy funkcji $y = 3x$ i $y = e^{x}$ przecinają się.

\subsection*{Rozwiązanie}
Zauważmy, że zadanie sprowadza się do znalezienia miejsca zerowego funkcji: $f(x) = 3x - e^{x}$.
\begin{figure}[H] 
    
    \centering 

    \includegraphics[width=0.7\textwidth, keepaspectratio]{../Examples/5funkja.png}

    \caption{Fragment wykresu funkcji $f(x) = 3x - e^{x}$}
    \label{fig:wykres-desmos}
    
\end{figure}
Na powyższym wykresie widzimy, że funkcja $f(x) = 3x - e^{x}$ ma dwa miejsca zerowe:
\begin{itemize}
    \item pierwsze miejsce zerowe znajduje się w przedziale $[0, 1]$
    \item drugie miejsce zerowe znajduje się w przedziale $[1, 2]$
\end{itemize}
Zastosujmy metodę bisekcji dla obu przedziałów z parametrami: $\delta = \frac{1}{2} \cdot 10^{-5}$ i $\epsilon = \frac{1}{2} \cdot 10^{-5}$, tak jak w poprzednim zadaniu.

\begin{table}[H]
    \centering
    \label{tab:bisekcja_dwa_przedzialy}
    \begin{tabular}{|l|c|c|c|c|c|}
    \hline
    \textbf{Opis} & \textbf{Przedział} & \textbf{Przybliżenie $x$} & \textbf{Wartość $f(x)$} & \textbf{Iter.} & \textbf{Err} \\ \hline
    Pierwiastek 1 & $[0.0, 1.0]$ & 0.619064331 & $3.48 \times 10^{-6}$ & 16 & 0 \\ \hline
    Pierwiastek 2 & $[1.0, 2.0]$ & 1.512134552 & $-5.29 \times 10^{-10}$ & 18 & 0 \\ \hline
    \end{tabular}
    \caption{Wyniki metody bisekcji dla dwóch różnych miejsc zerowych}

\end{table}

\subsection*{Wnioski}
Żeby rozwiązać ten problem, musieliśmy go przekształcić do postaci jednej funkcji, aby zastosowanie metody bisekcji było możliwe.
Metoda bisekcji zadziałała poprawnie dla obu przedziałów, zwracając kod błędu $err = 0$.
Pierwsze miejsce zerowe zostało znalezione w 16 iteracjach, a drugie w 18 iteracjach.
Wartości $f(x)$ dla znalezionych miejsc zerowych są bardzo bliskie zeru, co świadczy o poprawności uzyskanych wyników.

\newpage
\section*{Zadanie 6}
\subsection*{Cel zadania}
Celem zadania jest znalezienie miejsca zerowego funkcji:
\begin{itemize}
    \item $f_{1}(x) = e^{1-x} - 1$
    \item $f(x) = x \cdot e^{-x}$
\end{itemize}
za pomocą metody bisekcji, Newtona i siecznych z parametrami: $\delta = 10^{-5}$ i $\epsilon = 10^{-5}$.
Musimy również wybrać odpowiednie przedziały i przybliżenia początkowe.
Następnie musimy sprawdzić, co stanie się, gdy w metodzie Newtona dla $f_1$ wybierzemy $x_1 \in (1, \infty)$ 
a dla $f_2$ wybierzemy $x_0 > 1$. 
Należy również odpowiedzieć na pytanie, czy dla $f_2$ możemy wybrać $x_0 = 1$.

\subsection*{Rozwiązanie}
\textbf{Dla funkcji} $f_{1}(x) = e^{1-x} - 1$:
\begin{itemize}
    \item Metoda bisekcji: przedział $[0.0, 2.0]$
    \item Metoda Newtona: przybliżenie początkowe: $x_0 = 0.5$, pochodna funkcji: $f'_{1}(x) = -e^{1-x}$
    \item Metoda siecznych: przybliżenia początkowe: $x_0 = 0.0$, $x_1 = 2.0$
\end{itemize}

\begin{figure}[H] 
    
    \centering 

    \includegraphics[width=0.7\textwidth, keepaspectratio]{../Examples/funkcja61.png}

    \caption{Fragment wykresu funkcji $f(x) = e^{1-x} - 1$}
    \label{fig:wykres-desmos}
    
\end{figure}

\begin{table}[h]
    \centering
    \label{tab:wyniki_porownanie_2}
    \begin{tabular}{|l|c|c|c|c|}
    \hline
    \textbf{Metoda} & \textbf{Przybliżenie $x$} & \textbf{Wartość $f(x)$} & \textbf{Iter.} & \textbf{Err} \\ \hline
    Bisekcji & 1.000000000 & 0.0 & 1 & 0 \\ \hline
    Newtona & 0.999999999 & $1.12 \times 10^{-10}$ & 4 & 0 \\ \hline
    Siecznych & 1.000001760 & $-1.76 \times 10^{-6}$ & 6 & 0 \\ \hline
    \end{tabular}
    \caption{Wyniki poszukiwania pierwiastka $x \approx 1.0$}
\end{table}
\end{document}