\documentclass{article}

% --- PAKIETY ---
\usepackage[utf8]{inputenc}
\usepackage[T1]{fontenc}
\usepackage{lmodern}
\usepackage[polish]{babel}
\usepackage[margin=1in]{geometry}

% Pakiety matematyczne i do symboli
\usepackage{amsmath}
\usepackage{amsfonts}
\usepackage{amssymb}

% Pakiety do tabel i grafiki
\usepackage{booktabs} % Lepsza jakość tabel
\usepackage{graphicx}
\usepackage{float}

% Pakiety do formatowania i numeracji
\usepackage{siunitx} % Ładniejsze formatowanie liczb (np. 1.23e-04)
\sisetup{
    output-exponent-marker = e,
    bracket-numbers = false,
    group-separator = {\,}, % Separator tysięcy
    scientific-notation = true
}

% Inne
\usepackage{hyperref} % Linki w dokumencie
\hypersetup{
    colorlinks=true,
    linkcolor=blue,
    filecolor=magenta,      
    urlcolor=cyan,
    pdftitle={Obliczenia Naukowe - Lista 1},
    pdfauthor={Wojciech Typer},
}

% --- USTAWIENIA DOKUMENTU ---
\title{Sprawozdanie z Laboratorium\\Obliczenia Naukowe - Lista 1}
\author{Wojciech Typer}

\begin{document}
\maketitle
\section{Zdanie 1}

\subsection{Epsilon maszynowy (\textit{macheps})}

Epsilonem maszynowym \textit{macheps} nazywamy najmniejszą liczbę dodatnią taką, że w arytmetyce zmiennoprzecinkowej zachodzi \(1.0 + \text{macheps} > 1.0\). Jest to miara precyzji obliczeń, która określa odległość od liczby 1.0 do następnej reprezentowalnej liczby maszynowej. Im mniejszy epsilon, tym większa precyzja arytmetyki, co jest bezpośrednio związane z liczbą bitów przeznaczonych na mantysę w danym typie zmiennoprzecinkowym.

Poniżej przedstawiono porównanie wartości \textit{macheps} uzyskanych iteracyjnie, wartości zwracanych przez funkcję \texttt{eps()} w Julii oraz wartości zdefiniowanych w pliku nagłówkowym \texttt{float.h} kompilatora C (GCC 13).

\begin{table}[H]
\centering
\caption{Porównanie wartości epsilona maszynowego.}
\label{tab:epsilon}
\begin{tabular}{llll}
\toprule
\textbf{Typ danych} & \textbf{Wartość z \texttt{float.h} (GCC)} & \textbf{Wartość z \texttt{eps(T)} (Julia)} & \textbf{Wartość wyznaczona iteracyjnie} \\
\midrule
\texttt{Float16} & \num{9.7656e-4} & \num{0.000977} & \num{0.000977} \\
\texttt{Float32} & \num{1.19209290e-7} & \num{1.1920929e-7} & \num{1.1920929e-7} \\
\texttt{Float64} & \num{2.2204460492503131e-16} & \num{2.220446049250313e-16} & \num{2.220446049250313e-16} \\
\bottomrule
\end{tabular}
\end{table}

Jak widać w tabeli \ref{tab:epsilon}, wartości uzyskane eksperymentalnie są zgodne z wartościami referencyjnymi. 

\subsection{Najmniejsza dodatnia liczba maszynowa (\textit{eta})}

Liczba \textit{eta} (\(\eta\)) to najmniejsza dodatnia wartość, jaką można reprezentować w danym standardzie zmiennoprzecinkowym. Wartość ta jest związana z liczbami subnormalnymi (denormalizowanymi), które pozwalają na płynne "wypełnienie" luki między zerem a najmniejszą dodatnią liczbą znormalizowaną.

\begin{itemize}
    \item \textbf{Związek z \textit{MIN\textsubscript{sub}}}: Liczba \textit{eta} jest tożsama z \textit{MIN\textsubscript{sub}}, czyli najmniejszą możliwą do reprezentowania dodatnią liczbą subnormalną. W języku Julia wartość tę można uzyskać za pomocą funkcji \texttt{nextfloat(T(0.0))}.
    \item \textbf{Związek z \textit{MIN\textsubscript{nor}}}: Funkcja \texttt{floatmin(T)} zwraca najmniejszą dodatnią liczbę \textbf{znormalizowaną}, znaną jako \textit{MIN\textsubscript{nor}}. Jest to wartość większa od \textit{eta}.
\end{itemize}

Wartości \textit{eta} wyznaczone iteracyjnie (poprzez dzielenie 1.0 przez 2 aż do uzyskania 0.0) są zgodne z wynikami funkcji \texttt{nextfloat(T(0.0))}. Porównanie tych wartości przedstawiono w tabeli \ref{tab:eta}.

\begin{table}[H]
\centering
\caption{Porównanie wartości \textit{eta} (\(\eta\)).}
\label{tab:eta}
\begin{tabular}{lll}
\toprule
\textbf{Typ danych} & \textbf{Wartość z \texttt{nextfloat(T(0.0))}} & \textbf{Wartość wyznaczona iteracyjnie} \\
\midrule
\texttt{Float16} & \num{6.0e-8} & \num{6.0e-8} \\
\texttt{Float32} & \num{1.4e-45} & \num{1.0e-45} \\ 
\texttt{Float64} & \num{5.0e-324} & \num{5.0e-324} \\
\bottomrule
\end{tabular}
\end{table}


\subsection{Największa wartość skończona (\textit{MAX})}

Liczba \textit{MAX} to największa skończona wartość, jaką można zapisać w danym typie zmiennoprzecinkowym. Próba reprezentacji liczby większej niż \textit{MAX} prowadzi do uzyskania wartości nieskończonej (\texttt{Inf}). Doświadczalne wyznaczenie tej wartości polega na iteracyjnym mnożeniu liczby przez 2, aż do momentu, gdy stanie się ona nieskończona, a następnie cofnięciu ostatniej operacji.

\begin{table}[H]
\centering
\caption{Porównanie maksymalnych wartości zmiennoprzecinkowych.}
\label{tab:max}
\begin{tabular}{lll}
\toprule
\textbf{Typ danych} & \textbf{Wartość z \texttt{float.h} (GCC)} & \textbf{Wartość wyznaczona iteracyjnie} \\
\midrule
\texttt{Float16} & \num{65504.0} & \num{6.55e4} \\
\texttt{Float32} & \num{3.40282347e+38} & \num{3.4028235e38} \\
\texttt{Float64} & \num{1.7976931348623157e+308} & \num{1.7976931348623157e308} \\
\bottomrule
\end{tabular}
\end{table}
\section{Zadanie 2}
W tabeli poniżej znajdują się wartości epsilona maszynowego, obliczone metodą Kahana i te, zwrócone przez funkcję \texttt{eps()} w Julii.
\begin{table}[H]
\centering
\label{tab:kahan_comparison}
\begin{tabular}{lll}
\toprule
\textbf{Typ danych} & \textbf{Wartość z metody Kahana} & \textbf{Wartość z \texttt{eps(T)} (Julia)} \\
\midrule
\texttt{Float16} & \num{-0.000977} & \num{0.000977} \\
\texttt{Float32} & \num{1.1920929e-7} & \num{1.1920929e-7} \\
\texttt{Float64} & \num{-2.220446049250313e-16} & \num{2.220446049250313e-16} \\
\bottomrule
\end{tabular}
\end{table}

\section{Zadanie 3}



\end{document}