\documentclass{article}
\usepackage[utf8]{inputenc}
\usepackage[T1]{fontenc}
\usepackage{lmodern}
\usepackage[polish]{babel}
\usepackage{amsmath}
\usepackage{tikz}
\usepackage{algorithm}
\usepackage{algpseudocode}
\usepackage{hyperref}
\usepackage{float}
\usepackage{graphicx}
\usepackage{mathtools}
\usepackage{amsmath}
\usepackage{amsfonts}
\usepackage{amsmath}
\usepackage{amsmath}
\usepackage{booktabs}
\usepackage[margin=1in]{geometry}




\title{Obliczenia Naukowe - Lista 1}
\author{Wojciech Typer}
\date{}

\begin{document}
\maketitle

\section*{zadanie 1}
Wartości epsilona z kompilatora GCC13: 

\begin{enumerate}
    \item 16 bitów: 0.00097656
    \item 32 bity: 1.19209290e-07F
    \item 64 bity: 2.2204460492503131e-16
\end{enumerate}

Wartości epsilona wyznaczone eksperymentalnie: \\
\begin{enumerate}
    \item 16 bitów: 0.000977
    \item 32 bity: 1.1920929e-7
    \item 64 bity: 2.220446049250313e-16
\end{enumerate} 

\textbf{Epsilon maszynowy}  to najmniejsza liczba, taka, że gdy dodamy do niej 1.0 w wybranej arytmetyce, komputer rozróżni ją od 1.0. \\
\textbf{Precyzja arytmetyki} oznacza, z jaką dokładnością komputer reprezentuje liczby rzeczywiste - Epsilon jest bezpośrednią miarą tej precyzji - im mniejszy
epsilon, tym większa precyzja. \\ Epsilon zależy od liczby bitów w mantysie (części znaczącej liczby zmiennoprzecinkowej). \\


\end{document} 