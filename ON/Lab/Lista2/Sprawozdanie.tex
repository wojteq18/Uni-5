\documentclass{article}

\usepackage[utf8]{inputenc}
\usepackage[T1]{fontenc}
\usepackage{lmodern}
\usepackage[polish]{babel}
\usepackage[margin=1in]{geometry}

\usepackage{amsmath}
\usepackage{amsfonts}
\usepackage{amssymb}

\usepackage{booktabs} % Lepsza jakość tabel
\usepackage{graphicx}
\usepackage{float}

\usepackage{siunitx} 
\sisetup{
    output-exponent-marker = e,
    bracket-numbers = false,
    group-separator = {\,}, 
    scientific-notation = true
}

% Inne
\usepackage{hyperref} 
\hypersetup{
    colorlinks=true,
    linkcolor=blue,
    filecolor=magenta,      
    urlcolor=cyan,
    pdftitle={Obliczenia Naukowe - Lista 1},
    pdfauthor={Wojciech Typer},
}

\title{Sprawozdanie z Laboratorium\\Obliczenia Naukowe - Lista 2}
\author{Wojciech Typer}

\begin{document}
\maketitle
\section*{Zadanie 1}
\begin{center}
    \textbf{Cel zadania:} Celem zadania jest porównanie wyników z zadania 5 z Listy 1 oraz obecnych. W obydwu zadaniach obliczamy iloczyn skalarny tymi samymi metodami, jednak dane wejściowe w obydwu zadaniach nieco się różnią - sprawdzamy jak usunięcię dziesiątej cyfry po przecinku wpłynie na wyniki 
\end{center}    
\vspace{0.5cm}
Wyniki z zadania 5 z Listy 1:
\begin{table}[H]
\centering
\label{tab:sum_methods}
\begin{tabular}{lllS[table-format=1.18e-1]} % 'S' z siunitx do wyrównania liczb
\toprule
\textbf{Liczba bitów} & \textbf{Metoda} & \textbf{Wynik} \\
\midrule
\addlinespace % Dodatkowa przestrzeń dla czytelności
\multicolumn{3}{l}{\textbf{Precyzja 32-bitowa (\texttt{Float32})}} \\
\cmidrule(r){1-3}
& Metoda 1                & -0.4999443 \\
& Metoda 2     & -0.4543457 \\
& Metoda 3      & -0.5 \\
& Metoda 4      & -0.5 \\
\addlinespace
\multicolumn{3}{l}{\textbf{Precyzja 64-bitowa (\texttt{Float64})}} \\
\cmidrule(r){1-3}
& Metoda 1                 & 1.0251881368296672e-10 \\
& Metoda 2     & -1.5643308870494366e-10 \\
& Metoda 3     & 0.0 \\
& Metoda 4      & 0.0 \\
\bottomrule
\end{tabular}
\end{table} 
\vspace{0.5cm}
Wyniki z obecnego zadania:
\begin{table}[H]
\centering
\label{tab:sum_methods}
\begin{tabular}{lllS[table-format=1.18e-1]} % 'S' z siunitx do wyrównania liczb
\toprule
\textbf{Liczba bitów} & \textbf{Metoda} & \textbf{Wynik} \\
\midrule
\addlinespace % Dodatkowa przestrzeń dla czytelności
\multicolumn{3}{l}{\textbf{Precyzja 32-bitowa (\texttt{Float32})}} \\
\cmidrule(r){1-3}
& Metoda 1                & -0.4999443 \\
& Metoda 2     & -0.4543457 \\
& Metoda 3      & -0.5 \\
& Metoda 4      & -0.5 \\
\addlinespace
\multicolumn{3}{l}{\textbf{Precyzja 64-bitowa (\texttt{Float64})}} \\
\cmidrule(r){1-3}
& Metoda 1                 & -0.004296342739891585 \\
& Metoda 2     & -0.004296342998713953 \\
& Metoda 3     & -0.004296342842280865 \\
& Metoda 4      & -0.004296342842280865 \\
\bottomrule
\end{tabular}
\end{table} 
\vspace{0.5cm}
Użyte metody:

\begin{description}
    \item[\textbf{Metoda 1:}] Obliczanie "w przód":
    \[ \sum_{i=1}^{n} x_i y_i \]

    \item[\textbf{Metoda 2:}] Obliczanie "w tył":
    \[ \sum_{i=n}^{1} x_i y_i \]

    \item[\textbf{Metoda 3:}] Sumowanie osobno iloczynów dodatnich w porządku od największego do najmniejszego i osobno iloczynów ujemnych w porządku od najmniejszego do największego, a następnie dodanie obliczonych sum częściowych.

    \item[\textbf{Metoda 4:}] Przeciwnie do metody 3.
\end{description}
%\vspace{0.1cm}
\begin{center}
    \textbf{Wnioski: } Zauważmy, że dla arytemtyki 32-bitowej wyniki nie uległy zmianie - wynika to ze zbyt małej precyzji tej arytmetyki. Dla arytmetyki 64-bitowej możemy zauważyć duże rozbieżności w uzyskanych wynikach, pomimo tego, że zmiana wektora x jest bardzo niewielka - możemy zatem wysnuć wnioski, że algorytmy, z których skorzystaliśmy są bardzo wrażliwe na zmiany danych, co z kolei świadczy o tym, że obliczenie iloczynu skalarnego $x \cdot y$ jest źle uwarunkowane.
\end{center}
\newpage
\section*{Zadanie 2}
\begin{center}
    \textbf{Cel zadania: } Narysowanie wykresu funkcji: $f(x) = e^{x} ln(1 + e ^{-x})$ w co najmniej dwóch programach do wizualizacji oraz policzenie granicy: $\lim_{x \to \infty} f(x)$ oraz porównanie uzyskanego wyniku z wykresem funkcji.
\end{center}
\begin{figure}[H] 
    
    \centering 
    
    \includegraphics[width=0.8\textwidth, keepaspectratio]{chart1.png}
    
    \caption{Wykres 1, stworzony w programie Desmos.}
    \label{fig:wykres-desmos}
    
\end{figure}
\begin{figure}[H] 
    
    \centering 
    
    \includegraphics[width=1.0\textwidth, keepaspectratio]{chart2.png}
    
    \caption{Wykres 2, stworzony w programie Geogebra.}
    \label{fig:wykres-desmos}
    
\end{figure}
\vspace{0.2cm}
Obliczmy teraz granicę funkcji $f(x) = e^{x} \ln(1 + e^{-x})$ przy $x$ dążącym do nieskończoności:
\begin{align*}
    \lim_{x \to \infty} e^{x} \ln(1 + e^{-x}) 
    &= \lim_{x \to \infty} \frac{\ln(1 + e^{-x})}{e^{-x}} \\
    &\overset{H}{=} \lim_{x \to \infty} \frac{\frac{d}{dx} \left( \ln(1 + e^{-x}) \right)}{\frac{d}{dx} \left( e^{-x} \right)} \\
    &= \lim_{x \to \infty} \frac{\frac{1}{1 + e^{-x}} \cdot (-e^{-x})}{-e^{-x}} \\
    &= \lim_{x \to \infty} \frac{1}{1 + e^{-x}} \\
    &= \frac{1}{1 + 0} \\
    &= 1
\end{align*}
\vspace{0.2cm}
\begin{center}
    \textbf{Wnioski: } Zauważmy, że obliczona granica nie pokrywa się z uzyskanymi wykresami funkcji. Na wykresach wartość funkcji zdaje się dążyć do zera wraz ze wzrostem wartości $x$. Dzieje się tak dlatego, że dla dużych wartości $x$ wyrażenie $ln(1 + e^{-x})$ jest bardzo małe i podczas obliczeń numerycznych jest zaokrąglane do zera co powoduje, że wartość funkcji $f(x)$ jest również zaokrąglana do zera. Czynnik $e^x$ dla dużych wartości x jest bardzo duży, a mnożenie liczb różniących się wielkością rzędów jest obarczone bardzo dużym błędem, przez co użyte programy graficzne pokazują błędne wyniki.
\end{center}
\newpage
\section*{Zadanie 3}
\end{document}