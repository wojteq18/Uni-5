\documentclass{article}

\usepackage[utf8]{inputenc}
\usepackage[T1]{fontenc}
\usepackage{lmodern}
\usepackage[polish]{babel}
\usepackage[margin=1in]{geometry}

\usepackage{amsmath}
\usepackage{amsfonts}
\usepackage{amssymb}

\usepackage{booktabs} % Lepsza jakość tabel
\usepackage{graphicx}
\usepackage{float}

\usepackage{siunitx} 
\sisetup{
    output-exponent-marker = e,
    bracket-numbers = false,
    group-separator = {\,}, 
    scientific-notation = true
}

% Inne
\usepackage{hyperref} 
\hypersetup{
    colorlinks=true,
    linkcolor=blue,
    filecolor=magenta,      
    urlcolor=cyan,
    pdftitle={Obliczenia Naukowe - Lista 1},
    pdfauthor={Wojciech Typer},
}

\title{Sprawozdanie z Laboratorium\\Obliczenia Naukowe - Lista 2}
\author{Wojciech Typer}

\begin{document}
\maketitle
\section*{Zadanie 1}
\begin{center}
    \textbf{Cel zadania:} Celem zadania jest porównanie wyników z zadania 5 z Listy 1 oraz obecnych. W obydwu zadaniach obliczamy iloczyn skalarny tymi samymi metodami, jednak dane wejściowe w obydwu zadaniach nieco się różnią - sprawdzamy jak usunięcię dziesiątej cyfry po przecinku wpłynie na wyniki 
\end{center}    
\vspace{0.5cm}
Wyniki z zadania 5 z Listy 1:
\begin{table}[H]
\centering
\label{tab:sum_methods}
\begin{tabular}{lllS[table-format=1.18e-1]} % 'S' z siunitx do wyrównania liczb
\toprule
\textbf{Liczba bitów} & \textbf{Metoda} & \textbf{Wynik} \\
\midrule
\addlinespace % Dodatkowa przestrzeń dla czytelności
\multicolumn{3}{l}{\textbf{Precyzja 32-bitowa (\texttt{Float32})}} \\
\cmidrule(r){1-3}
& Metoda 1                & -0.4999443 \\
& Metoda 2     & -0.4543457 \\
& Metoda 3      & -0.5 \\
& Metoda 4      & -0.5 \\
\addlinespace
\multicolumn{3}{l}{\textbf{Precyzja 64-bitowa (\texttt{Float64})}} \\
\cmidrule(r){1-3}
& Metoda 1                 & 1.0251881368296672e-10 \\
& Metoda 2     & -1.5643308870494366e-10 \\
& Metoda 3     & 0.0 \\
& Metoda 4      & 0.0 \\
\bottomrule
\end{tabular}
\end{table} 
\vspace{0.5cm}
Wyniki z obecnego zadania:
\begin{table}[H]
\centering
\label{tab:sum_methods}
\begin{tabular}{lllS[table-format=1.18e-1]} % 'S' z siunitx do wyrównania liczb
\toprule
\textbf{Liczba bitów} & \textbf{Metoda} & \textbf{Wynik} \\
\midrule
\addlinespace % Dodatkowa przestrzeń dla czytelności
\multicolumn{3}{l}{\textbf{Precyzja 32-bitowa (\texttt{Float32})}} \\
\cmidrule(r){1-3}
& Metoda 1                & -0.4999443 \\
& Metoda 2     & -0.4543457 \\
& Metoda 3      & -0.5 \\
& Metoda 4      & -0.5 \\
\addlinespace
\multicolumn{3}{l}{\textbf{Precyzja 64-bitowa (\texttt{Float64})}} \\
\cmidrule(r){1-3}
& Metoda 1                 & -0.004296342739891585 \\
& Metoda 2     & -0.004296342998713953 \\
& Metoda 3     & -0.004296342842280865 \\
& Metoda 4      & -0.004296342842280865 \\
\bottomrule
\end{tabular}
\end{table} 
\vspace{0.5cm}
Użyte metody:

\begin{description}
    \item[\textbf{Metoda 1:}] Obliczanie "w przód":
    \[ \sum_{i=1}^{n} x_i y_i \]

    \item[\textbf{Metoda 2:}] Obliczanie "w tył":
    \[ \sum_{i=n}^{1} x_i y_i \]

    \item[\textbf{Metoda 3:}] Sumowanie osobno iloczynów dodatnich w porządku od największego do najmniejszego i osobno iloczynów ujemnych w porządku od najmniejszego do największego, a następnie dodanie obliczonych sum częściowych.

    \item[\textbf{Metoda 4:}] Przeciwnie do metody 3.
\end{description}
%\vspace{0.1cm}
\begin{center}
    \textbf{Wnioski: } Zauważmy, że dla arytemtyki 32-bitowej wyniki nie uległy zmianie - wynika to ze zbyt małej precyzji tej arytmetyki. Dla arytmetyki 64-bitowej możemy zauważyć duże rozbieżności w uzyskanych wynikach, pomimo tego, że zmiana wektora x jest bardzo niewielka - możemy zatem wysnuć wnioski, że algorytmy, z których skorzystaliśmy są bardzo wrażliwe na zmiany danych, co z kolei świadczy o tym, że obliczenie iloczynu skalarnego $x \cdot y$ jest źle uwarunkowane.
\end{center}
\newpage
\section*{Zadanie 2}
\begin{center}
    \textbf{Cel zadania: } Narysowanie wykresu funkcji: $f(x) = e^{x} ln(1 + e ^{-x})$ w co najmniej dwóch programach do wizualizacji oraz policzenie granicy: $\lim_{x \to \infty} f(x)$ oraz porównanie uzyskanego wyniku z wykresem funkcji.
\end{center}
\begin{figure}[H] 
    
    \centering 
    
    \includegraphics[width=0.8\textwidth, keepaspectratio]{chart1.png}
    
    \caption{Wykres 1, stworzony w programie Desmos.}
    \label{fig:wykres-desmos}
    
\end{figure}
\begin{figure}[H] 
    
    \centering 
    
    \includegraphics[width=1.0\textwidth, keepaspectratio]{chart2.png}
    
    \caption{Wykres 2, stworzony w programie Geogebra.}
    \label{fig:wykres-desmos}
    
\end{figure}
\vspace{0.2cm}
Obliczmy teraz granicę funkcji $f(x) = e^{x} \ln(1 + e^{-x})$ przy $x$ dążącym do nieskończoności:
\begin{align*}
    \lim_{x \to \infty} e^{x} \ln(1 + e^{-x}) 
    &= \lim_{x \to \infty} \frac{\ln(1 + e^{-x})}{e^{-x}} \\
    &\overset{H}{=} \lim_{x \to \infty} \frac{\frac{d}{dx} \left( \ln(1 + e^{-x}) \right)}{\frac{d}{dx} \left( e^{-x} \right)} \\
    &= \lim_{x \to \infty} \frac{\frac{1}{1 + e^{-x}} \cdot (-e^{-x})}{-e^{-x}} \\
    &= \lim_{x \to \infty} \frac{1}{1 + e^{-x}} \\
    &= \frac{1}{1 + 0} \\
    &= 1
\end{align*}
\vspace{0.2cm}
\begin{center}
    \textbf{Wnioski: } Zauważmy, że obliczona granica nie pokrywa się z uzyskanymi wykresami funkcji. Na wykresach wartość funkcji zdaje się dążyć do zera wraz ze wzrostem wartości $x$. Dzieje się tak dlatego, że dla dużych wartości $x$ wyrażenie $ln(1 + e^{-x})$ jest bardzo małe i podczas obliczeń numerycznych jest zaokrąglane do zera co powoduje, że wartość funkcji $f(x)$ jest również zaokrąglana do zera. Czynnik $e^x$ dla dużych wartości x jest bardzo duży, a mnożenie liczb różniących się wielkością rzędów jest obarczone bardzo dużym błędem, przez co użyte programy graficzne pokazują błędne wyniki.
\end{center}
\newpage
\section*{Zadanie 3}
\begin{center}
    \textbf{Cel zadania: } Rozwiązać układ równań liniowych postaci $Ax = b$ dla danej macierzy wspołczynników $A \in R^{n \times n}$ oraz wektora prawych stron $b \in R^{n}$, dwoma metodami: metodą eliminacji Gaussa, oraz metodą korzystającą wprost z równania $x = A^{-1}b$.
    Macierz A ma być wygenerowana na dwa sposoby: jako macierz Hilberta oraz jako macierz losowa o zadanym uwarunkowaniu.
\end{center}
\begin{table}[H]
    \centering
    \caption{Porównanie błędów względnych dla macierzy Hilberta o rosnącym rozmiarze $n$}
    \label{tab:hilbert_errors}
    % Ustawienia dla pakietu siunitx do formatowania liczb
    \sisetup{
        round-mode = figures,
        round-precision = 3,
        scientific-notation = true
    }
    \begin{tabular}{c S[table-format=1.2e2] c S[table-format=1.2e2] S[table-format=1.2e2]}
        \toprule
        % Nagłówki kolumn
        {$n$} & {cond(A)} & {rank(A)} & {$\frac{\|\mathbf{x} - \tilde{\mathbf{x}}_{\text{gauss}}\|}{\|\mathbf{x}\|}$} & {$\frac{\|\mathbf{x} - \tilde{\mathbf{x}}_{\text{inv}}\|}{\|\mathbf{x}\|}$} \\
        \midrule
        % Dane z Twoich wyników
        1  & 1.0e0   & 1 & 0.0e0    & 0.0e0    \\
        2  & 1.93e1  & 2 & 5.66e-16 & 1.40e-15 \\
        3  & 5.24e2  & 3 & 8.02e-15 & 0.0e0    \\
        4  & 1.55e4  & 4 & 4.14e-14 & 0.0e0    \\
        5  & 4.77e5  & 5 & 1.68e-12 & 3.35e-12 \\
        6  & 1.50e7  & 6 & 2.62e-10 & 2.02e-10 \\
        7  & 4.75e8  & 7 & 1.26e-8  & 4.71e-9  \\
        8  & 1.53e10 & 8 & 6.12e-8  & 3.08e-7  \\
        9  & 4.93e11 & 9 & 3.88e-6  & 4.54e-6  \\
        10 & 1.60e13 & 10& 8.67e-5  & 2.50e-4  \\
        11 & 5.22e14 & 10& 1.58e-4  & 7.62e-3  \\
        12 & 1.75e16 & 11& 1.34e-1  & 2.59e-1  \\
        13 & 3.19e18 & 11& 1.10e-1  & 5.33e0   \\
        14 & 6.20e17 & 11& 1.46e0   & 8.71e0   \\
        15 & 3.68e17 & 12& 4.70e0   & 7.34e0   \\
        16 & 7.05e17 & 12& 5.42e1   & 2.98e1   \\
        17 & 1.25e18 & 12& 1.37e1   & 1.05e1   \\
        18 & 2.25e18 & 12& 1.03e1   & 2.48e1   \\
        \bottomrule
    \end{tabular}
\end{table}

\begin{table}[H]
    \centering
    \caption{Porównanie błędów względnych dla losowych macierzy $R_n$ o zadanym wskaźniku uwarunkowania}
    \label{tab:random_errors}
    % Ustawienia dla pakietu siunitx
    \sisetup{
        round-mode = figures,
        round-precision = 3,
        scientific-notation = true
    }
    % POPRAWKA TUTAJ: Dodano @{\hspace{1.5em}} aby rozsunąć kolumny
    \begin{tabular}{
        S[table-format=1.0e1] @{\hspace{2.5em}}
        c                     @{\hspace{2.5em}}
        S[table-format=1.2e2] @{\hspace{2.5em}}
        c                     @{\hspace{2.5em}}
        S[table-format=1.2e-1] @{\hspace{2.5em}}
        S[table-format=1.2e-1]
    }
        \toprule
        % Nagłówki kolumn
        {$c$} & {$n$} & {cond(A)} & {rank(A)} & {$\frac{\|\mathbf{x} - \tilde{\mathbf{x}}_{\text{gauss}}\|}{\|\mathbf{x}\|}$} & {$\frac{\|\mathbf{x} - \tilde{\mathbf{x}}_{\text{inv}}\|}{\|\mathbf{x}\|}$} \\
        \midrule
        % n = 
        1e0   & 5 & 1.00e0  & 5 & 1.99e-16 & 1.40e-16 \\
        1e1   & 5 & 1.00e1  & 5 & 1.40e-16 & 1.49e-16 \\
        1e3   & 5 & 1.00e3  & 5 & 5.93e-14 & 5.90e-14 \\
        1e7   & 5 & 1.00e7  & 5 & 3.37e-10 & 3.51e-10 \\
        1e12  & 5 & 1.00e12 & 5 & 1.32e-5  & 1.35e-5  \\
        1e16  & 5 & 8.26e15 & 4 & 1.66e-1  & 1.71e-1  \\
        \midrule
        % n = 10
        1e0   & 10 & 1.00e0  & 10 & 2.67e-16 & 3.46e-16 \\
        1e1   & 10 & 1.00e1  & 10 & 5.04e-16 & 3.29e-16 \\
        1e3   & 10 & 1.00e3  & 10 & 1.52e-14 & 1.60e-14 \\
        1e7   & 10 & 1.00e7  & 10 & 1.53e-10 & 1.24e-10 \\
        1e12  & 10 & 1.00e12 & 10 & 1.74e-6  & 4.22e-6  \\
        1e16  & 10 & 8.25e15 & 9  & 2.13e-1  & 2.86e-1  \\
        \midrule
        % n = 20
        1e0   & 20 & 1.00e0  & 20 & 4.44e-16 & 4.82e-16 \\
        1e1   & 20 & 1.00e1  & 20 & 3.24e-16 & 4.04e-16 \\
        1e3   & 20 & 1.00e3  & 20 & 4.18e-15 & 4.62e-15 \\
        1e7   & 20 & 1.00e7  & 20 & 3.35e-10 & 4.61e-10 \\
        1e12  & 20 & 1.00e12 & 20 & 1.32e-5  & 1.04e-5  \\
        1e16  & 20 & 8.31e15 & 19 & 5.17e-2  & 3.46e-2  \\
        \bottomrule
    \end{tabular}
\end{table}
\newpage
\begin{center}
    \textbf{Wnioski: } W przypadku macierzy Hilberta, wraz ze wzrostem $n$ rośnie jej wskaźnik uwarunkowania, co przekłada się na wzrost błędów względnych obu metod rozwiązywania układu równań.
    Możemy z tego wywnioskować, że zadanie obliczenia układu równań z macierzą Hilberta jest źle uwarunkowane.
    W przypadku macierzy losowych, możemy zauważyć, że błędy względne są znacznie bardziej zależne od zadanych wskaźników uwarunkowania $c$ niż od rozmiaru
    macierzy $n$.
    Gdy dowolna macierz ma wysoki wskaźnik uwarunkowania, to zadanie obliczenia układu równań z tą macierzą jest źle uwarunkowane.
\end{center}  
\section*{Zadanie 4}
\begin{center}
    \textbf{Cel zadania: } Dany jest wielomian P, będący naturalną postacią wielomianu Wilkinsona: $p(x) = \prod_{i=1}^{20} (x - i)$. Należy obliczyć: $|P(z_{k})|, |p(z_{k})| oraz |z_{k} - k|$
    Następnie należy powtórzyć eksperyment Wilkinsona, tj. zmienić wpółczynik przy $x^{19} z -210$ na $-210 - 2^{-23}$ i opisać otrzymane wyniki
\end{center}  

\begin{table}[H]
    \centering
    \caption{Analiza błędów dla pierwiastków wielomianu Wilkinsona w jego nie zaburzonej reprezentacji numerycznej}
    \label{tab:wilkinson_ideal}
    
    \sisetup{
        scientific-notation = true,
        round-mode = figures,
        round-precision = 3
    }
    
    \begin{tabular}{
        c                   
        c                   
        S[table-format=1.4] 
        S[table-format=1.2e+2] 
        S[table-format=1.2e+2] 
    }
        \toprule
        {$k$} & {$z_k$} & {$\lvert z_k - k \rvert$} & {$\lvert P(z_k) \rvert$} & {$\lvert p(z_k) \rvert$} \\
        \midrule
        
        % POPRAWKA TUTAJ: Dodano nawiasy klamrowe {} wokół liczb w trzeciej kolumnie
        1  & 1.0      & {0.0000} & 3.57e4 & 3.66e4 \\
        2  & 2.0      & {0.0000} & 1.76e5 & 1.81e5 \\
        3  & 3.0      & {0.0000} & 2.79e5 & 2.90e5 \\
        4  & 4.0      & {0.0000} & 3.03e6 & 2.04e6 \\
        5  & 5.0      & {0.0000} & 2.29e7 & 2.09e7 \\
        6  & 6.0      & {0.0000} & 1.29e8 & 1.13e8 \\
        7  & 7.0001   & {0.0001} & 4.81e8 & 4.57e8 \\
        8  & 7.9994   & {0.0006} & 1.64e9 & 1.56e9 \\
        9  & 9.0029   & {0.0029} & 4.88e9 & 4.69e9 \\
        10 & 9.9904   & {0.0096} & 1.36e10 & 1.26e10 \\
        11 & 11.025   & {0.0250} & 3.59e10 & 3.30e10 \\
        12 & 11.9533  & {0.0467} & 7.53e10 & 7.39e10 \\
        13 & 13.0743  & {0.0743} & 1.96e11 & 1.85e11 \\
        14 & 13.9148  & {0.0852} & 3.58e11 & 3.55e11 \\
        15 & 15.0755  & {0.0755} & 8.22e11 & 8.42e11 \\
        16 & 15.9463  & {0.0537} & 1.55e12 & 1.57e12 \\
        17 & 17.0254  & {0.0254} & 3.69e12 & 3.32e12 \\
        18 & 17.9909  & {0.0091} & 7.65e12 & 6.34e12 \\
        19 & 19.0019  & {0.0019} & 1.14e13 & 1.23e13 \\
        20 & 19.9998  & {0.0002} & 2.79e13 & 2.32e13 \\
        
        \bottomrule
    \end{tabular}
\end{table}
\newpage
\begin{table}[H]
    \centering
    \caption{Analiza błędów dla pierwiastków zaburzonego wielomianu Wilkinsona}
    \label{tab:wilkinson_perturbed}
    
    \sisetup{
        output-complex-root = \text{i}, % Ustawia 'i' jako symbol części urojonej
        complex-root-position = after-number,
        scientific-notation = true,
        round-mode = figures,
        round-precision = 3
    }
    
    \begin{tabular}{
        c                 % Kolumna k
        c                 % Kolumna zk (ręcznie, bo zespolone)
        S[table-format=1.4] % Kolumna |zk - k|
        S[table-format=1.2e+2] % Kolumna |P(zk)|
        S[table-format=1.2e+2] % Kolumna |p(zk)|
    }
        \toprule
        {$k$} & {Przybliżony $z_k$} & {$\lvert z_k - k \rvert$} & {$\lvert P(z_k) \rvert$} & {$\lvert p(z_k) \rvert$} \\
        \midrule
        
        1  & 1.0                  & {0.0000} & 2.03e4 & 2.00e4 \\
        2  & 2.0                  & {0.0000} & 3.47e5 & 3.52e5 \\
        3  & 3.0                  & {0.0000} & 2.26e6 & 2.42e6 \\
        4  & 4.0                  & {0.0000} & 1.05e7 & 1.13e7 \\
        5  & 5.0                  & {0.0000} & 3.76e7 & 4.48e7 \\
        6  & 6.0                  & {0.0000} & 1.31e8 & 2.14e8 \\
        7  & 6.9996               & {0.0004} & 3.94e8 & 1.78e9 \\
        8  & 8.0078               & {0.0078} & 1.18e9 & 1.87e10 \\
        9  & 8.9158               & {0.0842} & 2.23e9 & 1.37e11 \\
        10 & 10.0955 - 0.6449i    & {0.6520} & 1.07e10 & 1.49e12 \\
        11 & 10.0955 + 0.6449i    & {1.1109} & 1.07e10 & 1.49e12 \\
        12 & 11.7939 - 1.6525i    & {1.6653} & 3.14e10 & 3.30e13 \\
        13 & 11.7939 + 1.6525i    & {2.0458} & 3.14e10 & 3.30e13 \\
        14 & 13.9924 - 2.5188i    & {2.5188} & 2.16e11 & 9.55e14 \\
        15 & 13.9924 + 2.5188i    & {2.7129} & 2.16e11 & 9.55e14 \\
        16 & 16.7307 - 2.8126i    & {2.9060} & 4.85e11 & 2.74e16 \\
        17 & 16.7307 + 2.8126i    & {2.8255} & 4.85e11 & 2.74e16 \\
        18 & 19.5024 - 1.9403i    & {2.4540} & 4.56e12 & 4.25e17 \\
        19 & 19.5024 + 1.9403i    & {2.0043} & 4.56e12 & 4.25e17 \\
        20 & 20.8469              & {0.8469} & 8.76e12 & 1.37e18 \\
        
        \bottomrule
    \end{tabular}
\end{table}
\begin{center}
    \textbf{Wnioski: } W przypadku nie zaburzonego wielomianu Wilkinsona, możemy zauważyć, że wyliczone pierwiastki nie pokrywają się z rzeczywistymi wartościami, 
    jednak błędy bezwzględne są stosunkowo niewielkie. Niemniej, wartości wielomianu (zarówno w jego naturalnej jak i oryginalnej formie) w tych punktach
    są bardzo duże i rosną wraz ze wzrostem wartości k. Dzieje się tak, ponieważ wyliczenie pierwiastków w wielomianie Wilkinsona jest ekstrymalnie źle uwarunkowane.

    W przypadku zaburzonego wielomianu Wilkinsona, sytuacja jest jeszcze gorsza - błędy bezwzględne pierwiastków są znacznie większe, a dodatkowo pojawiają się
    pierwiastki zespolone. Wszystkie te błędy wynikają z faktu, że jesteśmy zmuszni pracowac w arytmetyce z ograniczoną precyzją, a wielomian Wilkinsona jest przykładem wielomianu, którego pierwiastki są bardzo wrażliwe na nawet niewielkie zmiany współczynników.
\end{center}
\newpage
\section*{Zadanie 5}
\begin{center}
    \textbf{Cel zadania: } Rozpatrujemy następujące równanie rekurencyjne: $p_{n+1} = p_n + rp_n(1 - p_n)$, gdzie $r$ jest pewną stałą, $r(1 - p_n)$ jest czynnikiem wzrostu populacji
    a $p_0$ jest wielkością populacji, stanowiącą procent maksymalnej wielkości populacji dla danego stanu środowiska.
    W zadaniu należy przeprowadzić 40 iteracji dla $p_0 = 0.01$ i $r = 3$ w arytmetyce float32, następnie ponownie wykonać 40 iteracji, z niewielką modyfikacją,
    tj. wykonać 10 iteracji, zatrzymać, zastosować obcięcie wyniku, odrzucając cyfry po 3 miejscu po przecinku i kontynuować kolejne 30 iteracji. Należy porównać oba wyniki.
    Następnie należy wykonać kolejnych 40 iteracji dla danych $p_0 = 0.01$ i $r = 3$ w arytmetyce float32 i float64.
\end{center}
\begin{table}[H]
    \centering
    \caption{Wyniki 40 iteracji dla modelu logistycznego z $p_0=0.01$ i $r=3$}
    \label{tab:logistic_40_steps}
        % Ustawienia lokalne, aby uniknąć problemów z globalną konfiguracją
    \sisetup{
        scientific-notation = false, % Wyłączamy notację naukową
        table-format = 1.19        % Format: 1 cyfra przed, do 19 po przecinku
    }
    \begin{tabular}{
        |c|c| % Kolumna dla numeru iteracji
    }
        \toprule
        {Numer iteracji ($n$)} & {Wartość $p_n$} \\
        \midrule
        1  & 0.0397 \\
        2  & \num{0.15407173} \\
        3  & \num{0.5450726} \\
        4  & \num{1.2889781} \\
        5  & \num{0.1715188} \\
        6  & \num{0.5978191} \\
        7  & \num{1.3191134} \\
        8  & \num{0.056273222} \\
        9  & \num{0.21559286} \\
        10 & \num{0.7229306} \\
        11 & \num{1.3238364} \\
        12 & \num{0.037716985} \\
        13 & \num{0.14660022} \\  
        14 & \num{0.521926} \\ 
        15 & \num{1.2704837} \\ 
        16 & \num{0.2395482} \\ 
        17 & \num{0.7860428} \\
        18 & \num{1.2905813} \\
        19 & \num{0.16552472} \\
        20 & \num{0.5799036} \\
        21 & \num{1.3107498} \\
        22 & \num{0.088804245} \\
        23 & \num{0.3315584} \\
        24 & \num{0.9964407} \\
        25 & \num{1.0070806} \\
        26 & \num{0.9856885} \\
        27 & \num{1.0280086} \\
        28 & \num{0.9416294} \\
        29 & \num{1.1065198} \\
        30 & \num{0.7529209} \\
        31 & \num{1.3110139} \\
        32 & \num{0.0877831} \\
        33 & \num{0.3280148} \\
        34 & \num{0.9892781} \\
        35 & \num{1.021099} \\
        36 & \num{0.95646656} \\
        37 & \num{1.0813814} \\
        38 & \num{0.81736827} \\
        39 & \num{1.2652004} \\
        40 & \num{0.25860548} \\
        \bottomrule
    \end{tabular}
\end{table}

\newpage

\begin{table}[H]
    \centering
    \caption{Wyniki 40 iteracji dla modelu logistycznego z $p_0=0.01$ i $r=3$, z obcięciem po 10 iteracjach}
    \label{tab:logistic_40_steps}
        % Ustawienia lokalne, aby uniknąć problemów z globalną konfiguracją
    \sisetup{
        scientific-notation = false, % Wyłączamy notację naukową
        table-format = 1.19        % Format: 1 cyfra przed, do 19 po przecinku
    }
    \begin{tabular}{
        |c|c| % Kolumna dla numeru iteracji
    }
        \toprule
        {Numer iteracji ($n$)} & {Wartość $p_n$} \\
        \midrule
        1  & 0.0397 \\
        2  & \num{0.15407173} \\
        3  & \num{0.5450726} \\
        4  & \num{1.2889781} \\
        5  & \num{0.1715188} \\
        6  & \num{0.5978191} \\
        7  & \num{1.3191134} \\
        8  & \num{0.056273222} \\
        9  & \num{0.21559286} \\
        10 & \num{0.7229306} \\
        \textbf{10} & \textbf{0.722} \\ % Obcięcie do 3 miejsc po przecinku
        11 & \num{1.3241479} \\
        12 & \num{0.036488414} \\
        13 & \num{0.14195944} \\  
        14 & \num{0.50738037} \\ 
        15 & \num{1.2572169} \\ 
        16 & \num{0.28708452} \\ 
        17 & \num{0.9010855} \\
        18 & \num{1.1684768} \\
        19 & \num{0.577893} \\
        20 & \num{1.3096911} \\
        21 & \num{0.09289217} \\
        22 & \num{0.34568182} \\
        23 & \num{1.0242395} \\
        24 & \num{0.94975823} \\
        25 & \num{1.0929108} \\
        26 & \num{0.7882812} \\
        27 & \num{1.2889631} \\
        28 & \num{0.17157483} \\
        29 & \num{0.59798557} \\
        30 & \num{1.3191822} \\
        31 & \num{0.05600393} \\
        32 & \num{0.21460639} \\
        33 & \num{0.7202578} \\
        34 & \num{1.3247173} \\
        35 & \num{0.034241438} \\
        36 & \num{0.13344833} \\
        37 & \num{0.48036796} \\
        38 & \num{1.2292118} \\
        39 & \num{0.3839622} \\
        40 & \num{1.093568} \\
        \bottomrule
    \end{tabular}
\end{table}

\begin{table}[H]
    \centering
    \caption{Wyniki 40 iteracji dla modelu logistycznego w arytmetyce Float64}
    \label{tab:logistic_new_data}
    
    % Ustawienia lokalne, aby uniknąć problemów z globalną konfiguracją
    \sisetup{
        scientific-notation = false, % Wyłączamy notację naukową
        table-format = 1.19        % Format: 1 cyfra przed, do 19 po przecinku
    }
    
    \begin{tabular}{
        |c|c|
    }
        \toprule
        {Numer iteracji ($n$)} & {Wartość $p_n$} \\
        \midrule
        1  & 0.0397 \\
        2  & 0.15407173000000002 \\
        3  & 0.5450726260444213 \\
        4  & 1.2889780011888006 \\
        5  & 0.17151914210917552 \\
        6  & 0.5978201201070994 \\
        7  & 1.3191137924137974 \\
        8  & 0.056271577646256565 \\
        9  & 0.21558683923263022 \\
        10 & 0.722914301179573 \\
        11 & 1.3238419441684408 \\
        12 & 0.03769529725473175 \\
        13 & 0.14651838271355924 \\
        14 & 0.521670621435246 \\
        15 & 1.2702617739350768 \\
        16 & 0.24035217277824272 \\
        17 & 0.7881011902353041 \\
        18 & 1.2890943027903075 \\
        19 & 0.17108484670194324 \\
        20 & 0.5965293124946907 \\
        21 & 1.3185755879825978 \\
        22 & 0.058377608259430724 \\
        23 & 0.22328659759944824 \\
        24 & 0.7435756763951792 \\
        25 & 1.315588346001072 \\
        26 & 0.07003529560277899 \\
        27 & 0.26542635452061003 \\
        28 & 0.8503519690601384 \\
        29 & 1.2321124623871897 \\
        30 & 0.37414648963928676 \\
        31 & 1.0766291714289444 \\
        32 & 0.8291255674004515 \\
        33 & 1.2541546500504441 \\
        34 & 0.29790694147232066 \\
        35 & 0.9253821285571046 \\
        36 & 1.1325322626697856 \\
        37 & 0.6822410727153098 \\
        38 & 1.3326056469620293 \\
        39 & 0.0029091569028512065 \\
        40 & 0.011611238029748606 \\
        \bottomrule
    \end{tabular}
\end{table}

\begin{center}
    \textbf{Wnioski: } W przypadku arytmetyki Float32 widzimy, że obcięcie wartości po 10 iteracjach prowadzi do znacznie różnych wyników końcowych.
    Oznacza to, że model logistyczny jest bardzo wrażliwy na dokładność obliczeń, a nawet niewielkie zmiany mogą prowadzić do znacznych różnic w wynikach.
\end{center}
\end{document}