\documentclass{article}

\usepackage[utf8]{inputenc}
\usepackage[T1]{fontenc}
\usepackage{lmodern}
\usepackage[polish]{babel}
\usepackage[margin=1in]{geometry}

\usepackage{amsmath}
\usepackage{amsfonts}
\usepackage{amssymb}

\usepackage{booktabs} % Lepsza jakość tabel
\usepackage{graphicx}
\usepackage{float}

\usepackage{siunitx} 
\sisetup{
    output-exponent-marker = e,
    bracket-numbers = false,
    group-separator = {\,}, 
    scientific-notation = true
}

% Inne
\usepackage{hyperref} 
\hypersetup{
    colorlinks=true,
    linkcolor=blue,
    filecolor=magenta,      
    urlcolor=cyan,
    pdftitle={Obliczenia Naukowe - Lista 1},
    pdfauthor={Wojciech Typer},
}

\title{Sprawozdanie z Laboratorium\\Obliczenia Naukowe - Lista 1}
\author{Wojciech Typer}

\begin{document}
\maketitle
\section*{Zadanie 1}
\begin{center}
    \textbf{Cel zadania:} Wyznaczyć ograniczenie wartości x, aby obliczenie różnicy: $y = \sqrt{x^2 + 1} - 1$ powodowało zmniejszenie dokładności najwyżej o 2 bity
\end{center}
Do rozwiązania zadania użyję następującego twierdzenia z wykładu: 
\begin{center}
    Jeśli x i y są dodatnimi liczbami w dwójkowej arytmetyce fl, takimi, że x > y i $2^{-q} \leq 1 - \frac{y}{x} \leq 2^{-p}$, to obliczenie różnicy fl(x - y) powoduje utratę dokładności o co najmniej p bitów i co najwyżej q bitów.
\end{center}
W maszym przypadku: $x = \sqrt{x^2 + 1}$ oraz $y = 1$. Zatem:
\begin{equation*}
    2^{-q} \leq 1 - \frac{1}{\sqrt{x^2 + 1}} 
\end{equation*}
Po przekształceniach otrzymujemy:
\begin{equation*}
    \frac{1}{(1 - 2^{-q})^2} - 1 \leq x^2 
\end{equation*}
Podstawiając q = 2 (dla podwójnej precyzji) otrzymujemy:
\begin{equation*}
    \frac{1}{(1 - 2^{-2})^2} - 1 \leq x^2 
\end{equation*}
Zatem ostatecznie:
\begin{equation*}
    \sqrt{\frac{1}{(1 - 2^{-2})^2} - 1} \leq |x| 
\end{equation*}
Po obliczeniach otrzymujemy:
\begin{equation*}
    |x| \geq \sqrt{ \frac{7}{9} } \approx 0.8815
\end{equation*}
\textbf{Wnioski: } Jeżeli chcemy, aby obliczenie różnicy y = $\sqrt{x^2 + 1} - 1$ nie powodowało zmniejszenia dokładności o więcej niż 2 bity, to musimy mieć $|x| \geq \sqrt{ \frac{7}{9} } \approx 0.8815$.

\newpage
\section*{Zadanie 2}
\begin{center}
    \textbf{Cel zadania:} Obliczyć, o ile bitów zmniejsza się dokładność różnicy $1 - cos(x)$ dla x = $\frac{1}{2}$
\end{center}  
Korzystając z twierdzenia użytego w poprzednim zadaniu:
\begin{equation*}
    2^{-q} \leq 1 - \frac{cos(x)}{1}
\end{equation*}  
Do rozwiązania powyższego równania użyjemy logarytmów o podstawie 2:
\begin{equation*}
    -q \approx log_2(1 - cos(x))
\end{equation*}
Przyjmijmy: $1 - cos(x) \approx 0.12241744$
\begin{equation*}
    -q \approx log_2(0.12241744)
\end{equation*}
Po obliczeniach otrzymujemy:
\begin{equation*}
    q \approx 3.0301
\end{equation*}
\textbf{Wnioski: } Obliczenie różnicy $1 - cos(x)$ dla x = $\frac{1}{2}$ powoduje zmniejszenie dokładności o około 3 bity.

\end{document}