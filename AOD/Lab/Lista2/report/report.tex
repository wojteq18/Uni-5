\documentclass{article}
\usepackage[utf8]{inputenc}
\usepackage[T1]{fontenc}
\usepackage{lmodern}
\usepackage[polish]{babel}
\usepackage{amsmath}
\usepackage{tikz}
\usepackage{algorithm}
\usepackage{algpseudocode}
\usepackage{hyperref}
\usepackage{float}
\usepackage{graphicx}
\usepackage{mathtools}
\usepackage{amsmath}
\usepackage{amsfonts}
\usepackage{amsmath}
\usepackage{amsmath}
\usepackage{booktabs}
\usepackage[margin=1in]{geometry}
\usepackage{amsbsy}
\usepackage{amsmath}



\title{Algorytmy Optymalizacji Dyskretnej - Lista 2}
\author{Wojciech Typer}
\date{}

\begin{document}
\maketitle

\section*{Zadanie 1}

\subsection*{Cel zadania}
Celem zadania jest zminimalizowanie kosztów dostawy paliwa

\subsection*{Opis modelu}
W zadaniu mamy następujące dane:
\begin{itemize}
    \item supply: $s_{i}$ ilość paliwa dostępna u dostawcy $i$
    \item demand: $d_{j}$ ilość paliwa potrzebna na stacji $j$
    \item cost: $c_{ij}$ koszt dostarczenia jednostki paliwa od dostawcy $i$ do stacji $j$
    \item Niech $S$ oznacza zbiór dostawców, a $D$ zbiór odbiorców paliwa
\end{itemize}

W modelu mamy następujące zmienne decyzyjne:
\begin{itemize}
    \item $x_{ij}$ - ilość paliwa dostarczona z magazynu $i$ do stacji $j$
\end{itemize}

\textbf{Ograniczenia:}
\begin{itemize}
    \item Podaż dla każdego dostawcy nie może zostać przekroczona:
    \begin{equation}
        \sum_{j \in D} x_{ij} \leq s_{i} \quad \forall i \in S
    \end{equation}
    \item Popyt dla każdej stacji musi zostać zaspokojony:
    \begin{equation}
        \sum_{i \in S} x_{ij} = d_{j} \quad \forall j \in D
    \end{equation}
    \item Zmienne decyzyjne nie mogą być ujemne:
    \begin{equation}
        x_{ij} \geq 0 \quad \forall i \in S, j \in D
    \end{equation}
    \item Aby model był możliwy do rozwiązania, całkowita podaż musi być większa bądź równa całkowitemu popytowi:
    \begin{equation}
        \sum_{i \in S} s_{i} \geq \sum_{j \in D} d_{j}
    \end{equation}
\end{itemize}

\textbf{Funkcja celu:} 
Chcemy zminimalizować całkowity koszt dostawy paliwa do odbiroców, tak aby każdy z nich otrzymał wymaganą ilość paliwa:
\begin{equation}
    \text{min}  \sum_{i \in S} \sum_{j \in D} c_{ij} x_{ij}
\end{equation}

\subsection*{Opis rozwiązania}
Wyniki możemy przedstawić za pomocą macierzy: 
\[
\begin{bmatrix}
0 & 165000.0 & 0 & 110000.0 \\
110000.0 & 55000.0 & 0 & 0 \\
0 & 0 & 330000.0 & 330000.0
\end{bmatrix}
\]
Którą należy interpretować w następujący sposób:
\begin{itemize}
    \item Firma 1 dostarcza 165000 jednostek paliwa do stacji 2 oraz 110000 jednostek do stacji 4, czyli wysyła w sumie 275000 jednostek paliwa
    \item Firma 2 dostarcza 110000 jednostek paliwa do stacji 1 oraz 55000 jednostek do stacji 2, czyli wysyła w sumie 165000 jednostek paliwa
    \item Firma 3 dostarcza 330000 jednostek paliwa do stacji 3 oraz 330000 jednostek do stacji 4, czyli wysyła w sumie 660000 jednostek paliwa
\end{itemize}
Zatem całkowity koszt optymalnego dostarczenia paliwa wynosi 8 525 000 jednostek waluty.
Z otrzymanych wyników możemy również wywnioskować, że wszystkie firmy dostarczają paliwo, oraz, że możliwości dostaw firmy 1 i 3 są w pełni wykorzystane.
\end{document} 