\documentclass{article}
\usepackage[utf8]{inputenc}
\usepackage[T1]{fontenc}
\usepackage{lmodern}
\usepackage[polish]{babel}
\usepackage{amsmath}
\usepackage{tikz}
\usepackage{algorithm}
\usepackage{algpseudocode}
\usepackage{hyperref}
\usepackage{float}
\usepackage{graphicx}
\usepackage{mathtools}
\usepackage{amsmath}
\usepackage{amsfonts}
\usepackage{amsmath}
\usepackage{amsmath}
\usepackage{booktabs}
\usepackage[margin=1in]{geometry}
\usepackage{amsbsy}
\usepackage{amsmath}



\title{Algorytmy Optymalizacji Dyskretnej - Lista 2}
\author{Wojciech Typer}
\date{}

\begin{document}
\maketitle

\section*{Zadanie 1}

\subsection*{Cel zadania}
Celem zadania jest zminimalizowanie kosztów dostawy paliwa

\subsection*{Opis modelu}
W zadaniu mamy następujące dane:
\begin{itemize}
    \item supply: $s_{i}$ ilość paliwa dostępna u dostawcy $i$
    \item demand: $d_{j}$ ilość paliwa potrzebna na stacji $j$
    \item cost: $c_{ij}$ koszt dostarczenia jednostki paliwa od dostawcy $i$ do stacji $j$
    \item Niech $S$ oznacza zbiór dostawców, a $D$ zbiór odbiorców paliwa
\end{itemize}

W modelu mamy następujące zmienne decyzyjne:
\begin{itemize}
    \item $x_{ij}$ - ilość paliwa dostarczona z magazynu $i$ do stacji $j$
\end{itemize}

\textbf{Ograniczenia:}
\begin{itemize}
    \item Podaż dla każdego dostawcy nie może zostać przekroczona:
    \begin{equation}
        \sum_{j \in D} x_{ij} \leq s_{i} \quad \forall i \in S
    \end{equation}
    \item Popyt dla każdej stacji musi zostać zaspokojony:
    \begin{equation}
        \sum_{i \in S} x_{ij} = d_{j} \quad \forall j \in D
    \end{equation}
    \item Zmienne decyzyjne nie mogą być ujemne:
    \begin{equation}
        x_{ij} \geq 0 \quad \forall i \in S, j \in D
    \end{equation}
    \item Aby model był możliwy do rozwiązania, całkowita podaż musi być większa bądź równa całkowitemu popytowi:
    \begin{equation}
        \sum_{i \in S} s_{i} \geq \sum_{j \in D} d_{j}
    \end{equation}
\end{itemize}

\textbf{Funkcja celu:} 
Chcemy zminimalizować całkowity koszt dostawy paliwa do odbiroców, tak aby każdy z nich otrzymał wymaganą ilość paliwa:
\begin{equation}
    \text{min}  \sum_{i \in S} \sum_{j \in D} c_{ij} x_{ij}
\end{equation}

\subsection*{Opis rozwiązania}
Wyniki możemy przedstawić za pomocą macierzy: 
\[
\begin{bmatrix}
0 & 165000.0 & 0 & 110000.0 \\
110000.0 & 55000.0 & 0 & 0 \\
0 & 0 & 330000.0 & 330000.0
\end{bmatrix}
\]
Którą należy interpretować w następujący sposób:
\begin{itemize}
    \item Firma 1 dostarcza 165000 jednostek paliwa do stacji 2 oraz 110000 jednostek do stacji 4, czyli wysyła w sumie 275000 jednostek paliwa
    \item Firma 2 dostarcza 110000 jednostek paliwa do stacji 1 oraz 55000 jednostek do stacji 2, czyli wysyła w sumie 165000 jednostek paliwa
    \item Firma 3 dostarcza 330000 jednostek paliwa do stacji 3 oraz 330000 jednostek do stacji 4, czyli wysyła w sumie 660000 jednostek paliwa
\end{itemize}
Zatem całkowity koszt optymalnego dostarczenia paliwa wynosi 8 525 000 jednostek waluty.
Z otrzymanych wyników możemy również wywnioskować, że wszystkie firmy dostarczają paliwo, oraz, że możliwości dostaw firmy 1 i 3 są w pełni wykorzystane.

\section*{Zadanie 2}
\subsection*{Cel zadania}
Wyznaczenie optymalnego tygodniowego planu produkcji poszczególnych wyrobów oraz obliczenie zysku z ich sprzedaży.

\subsection*{Opis modelu}
W zadaniu mamy następujące dane:
\begin{itemize}
    \item Zbiór maszyn produkcyjnych $M$
    \item Zbiór gotowych produktów $P$
    \item Ceny produktów $p_{i}$ dla każdego produktu $i \in P$
    \item Koszty pracy maszyn (za godzinę) $c_{m}$ dla każdej maszyny $m \in M$
    \item Koszty materiałów $k_{i}$ dla każdego produktu $i \in P$
    \item Maksymalny tygodniowy popyt $d_{i}$ dla każdego produktu $i \in P$
    \item Czas produkcji jednostki produktu $i$ na maszynie $m$ wynosi $t_{mi}$
    \item Dostępny czas pracy maszyny $m \in M$ w tygodniu wynosi $T_{m}$
\end{itemize} 
W modelu mamy następujące zmienne decyzyjne:
\begin{itemize}
    \item Ilość wyprodukowanego produktu $i \in P$ oznaczona jako $x_{i}$
\end{itemize}   
\subsubsection*{Ograniczenia:}
\begin{itemize}
    \item Dostępny czas pracy maszyny nie może zostać przekroczony:
    \begin{equation}
        \sum_{i \in P} t_{mi} x_{i} \leq T_{m} \quad \forall m \in M
    \end{equation}
    \item Produkcja nie może przekroczyć maksymalnego popytu oraz musi być nieujemna:
    \begin{equation}
        0 \leq x_{i} \leq d_{i} \quad \forall i \in P
    \end{equation}
\end{itemize}
\subsubsection*{Funkcja celu:}
Celem jest maksymalizacja zysku z produkcji wyrobów:
\begin{equation}
    \text{max} \sum_{i \in P} (p_{i} - k_{i}) x_{i} - \sum_{m \in M} c_{m} \left( \sum_{i \in P} t_{mi} x_{i} \right)
\end{equation}
\subsection*{Opis rozwiązania}
Otrzymujemy następujące rozwiązanie: 
\begin{equation}
    X = [125.0, 100.0, 150.0, 500.0] \text{, gdzie } X = [x_{i}]
\end{equation}  
Jest to wektor wyprodukowanych jednostek wyrobów $i \in P$. Uzyskany w ten sposób zysk wynosi 3632.5 dolarów. Czas pracy maszyn wynosi: 
\begin{itemize}
    \item Maszyna 1: 3520 minut
    \item Maszyna 2: 3600 minut
    \item Maszyna 3: 2100 minut
\end{itemize}  

\section*{Zadanie 3}
\subsection*{Cel zadania}
Wyznaczyć plan produkcji i magazynowania wytwarzanego towaru, który spełnia zapotrzebowanie w każdym okresie i minimalizuje łączny koszt.

\subsection*{Opis modelu}
W zadaniu mamy dostępne następujące dane:
\begin{itemize}
    \item $T$ - liczba okresów planowania
    \item $c_{j}$ - koszt produkcji jednostki towaru w okresie $j \in T$
    \item $o_{j}$ - koszt produkcji jednostki towaru w nadprodukcji w okresie $j \in T$
    \item $a_{j}$ - ilość możliwej nadprodukcji w okresie $j \in T$
    \item $d_{j}$ - zapotrzebowanie na towar w okresie $j \in T$
    \item $P_{max}$ - maksymalna ilość towaru, którą firma może wyprodukować w okresie czasu
    \item $h$ - koszt magazynowania jednostki towaru przez jeden okres czasu
    \item $s_{0}$ - początkowa ilość towaru w magazynie
    \item $S_{max}$ - maksymalna pojemność magazynu
\end{itemize}
W modelu mamy następujące zmienne decyzyjne:
\begin{itemize}
    \item $x_{j}$ - ilość produkowanego towaru w okresie $j \in T$ w standardowej produkcji
    \item $y_{j}$ - ilość produkowanego towaru w okresie $j \in T$ w nadprodukcji
    \item $s_{j}$ - ilość towaru w magazynie na koniec okresu $j \in T$
\end{itemize}

\newpage
\subsection*{Ograniczenia}
\begin{itemize}
    \item Nie ponadmiarowa produkcja nie może przekroczyć maksymalnej produkcji:
    \begin{equation}
        0 \leq x_{j} \leq P_{max} \quad \forall j \in T
    \end{equation}
    \item Ponadmiarowa produkcja nie może przekroczyć możliwej nadprodukcji:
    \begin{equation}
        0 \leq y_{j} \leq a_{j} \quad \forall j \in T
    \end{equation}
    \item Ilość towaru w magazynie nie może przekroczyć jego maksymalnej pojemności:
    \begin{equation}
        0 \leq s_{j} \leq S_{max} \quad \forall j \in T
    \end{equation}
    \item Bilans zapasu magazynowego w każdym okresie musi zostać zachowany:
    \begin{equation}
        s_{j-1} + x_{j} + y{j} = d_{j} + s_{j} \quad \forall j \in T
    \end{equation} 
\end{itemize}

\subsection*{Funkcja celu}
Celem jest minimalizacja całkowitych kosztów produkcji i magazynowania towaru:
\begin{equation}
    \text{min} \sum_{j \in T} c_{j} x_{j} + o_{j} y_{j} + h s_{j}
\end{equation}

\subsection*{Opis rozwiązania}
Otrzymujemy następujące wyniki:
\begin{itemize}
    \item Okres 1: $x_{1} = 100, y_{1} = 15, s_{1} = 0$
    \item Okres 2: $x_{2} = 100, y_{2} = 50, s_{2} = 70$
    \item Okres 3: $x_{3} = 100, y_{3} = 0, s_{3} = 45$
    \item Okres 4: $x_{4} = 100, y_{4} = 50, s_{4} = 0$
\end{itemize}
Na podstawie wyników możemy stwierdzić, że minimalny koszt produkcji i magazynowania wynosi 3 842 500 dolarów, 
w okresach 1, 2 i 4 firma korzysta z nadprodukcji a możliwości magazynowania towaru są wyczerpane na koniec drugiego okresu.

\newpage
\section*{Zadanie 4}
\subsection*{Cel zadania}
Celem jest znalezienie połączenia z miasta $i$ do miasta $j$, którego całkowity koszt jest najmniejszy i całkowity czas przejazdu nie przekracza z góry zadanego czasu $T$.

\subsection*{Opis modelu}
W zadaniu mamy następujące dane:
\begin{itemize}
    \item $N$ - zbiór miasta
    \item $A$ - zbiór dróg, gdzie każda droga $(i, j)$ łączy miasto $i \in N$ z miastem $j \in T$
    \item $c_{ij}$ - koszt przejazdu drogą $(i, j) \in A$
    \item $t_{ij}$ - czas przejazdu drogą $(i, j) \in A$
    \item $i$ - miasto początkowe
    \item $j$ - miasto końcowe
    \item $T$ - maksymalny dozwolony czas podróży
\end{itemize}
W modelu mamy następujące zmienne decyzyjne:
\begin{itemize}
    \item $x_{ij}$ - zmienna binarna wskazująca czy droga $(i, j) \in A$ jest w wybrana w trasie
\end{itemize}

\subsection*{Ograniczenia}
\begin{itemize}
    \item Całkowity czas podróży nie może przekroczyć zadanego $T$:
    \begin{equation}
        \sum_{(i, j) \in A} t_{ij} x_{ij} \leq T
    \end{equation}
    \item Zachowanie przepływu w każdym mieście $k \in N$:
        \[
            \sum_{j: (k,j) \in A} x_{kj} - \sum_{i: (i,k) \in A} x_{ik} =
            \begin{cases}
                1, & \text{jeśli } k = i^{\circ} \\
                -1, & \text{jeśli } k = j^{\circ} \\
                0 & \text{w przeciwnym razie}
            \end{cases}
        \]
    \item Zmienne decyzyjne muszą być binarne:
    \begin{equation}
        x_{ij} \in \{0, 1\} \quad \forall (i, j) \in A
    \end{equation}
\end{itemize}

\subsection*{Funkcja celu}
Celem jest minimalizacja kosztu przejazdu z miasta początkowego do miasta końcowego:
\begin{equation}
    \text{min} \sum_{(i, j) \in A} c_{ij} x_{ij} 
\end{equation}

\subsection*{Opis rozwiązań}
Dla danych z zadania, to jest:
\begin{itemize}
    \item $N$ = 10
    \item $i$ = 1
    \item $j$ = 10
    \item $T_{max}$ = 15
    \item $arcs$ - podane 
\end{itemize}
Otrzymuejmy następujące wyniki:
\begin{itemize}
    \item Minimalny koszt: 13.0
    \item Czas przejazdu: 15.0
    \item Ścieżka: $1 \rightarrow 2 \rightarrow 3 \rightarrow 5 \rightarrow 7 \rightarrow 9 \rightarrow 10$
\end{itemize}
Dla danych z własnego egzemplarza, to jest: 
\begin{itemize}
    \item $N$ = 12
    \item $i$ = 1
    \item $j$ = 9
    \item $T_{max}$ = 30
    \item $arcs$ - jak na rysunku poniżej
\end{itemize}

\begin{figure}[H] 
    
    \centering 

    \includegraphics[width=0.9\textwidth, keepaspectratio]{/home/wojteq18/sem5/AOD/Lab/Lista2/examples/my_graph.png}

    \caption{Własny egzemplarz do zadania 4}
    
\end{figure}
\newpage
Otrzymujemy następujące wyniki:
\begin{itemize}
    \item Minimalny koszt: 10.0
    \item Czas przejazdu: 29.0
    \item Ścieżka: $1 \rightarrow 12 \rightarrow 3 \rightarrow 9$
\end{itemize}
Zauważmy, że ograniczenie na całkowitoliczbowość jest istotne. Rozpatrujemy następujący graf:
\begin{figure}[H] 
    
    \centering 

    \includegraphics[width=0.9\textwidth, keepaspectratio]{/home/wojteq18/sem5/AOD/Lab/Lista2/examples/bad_graph.png}

    \caption{Przykład grafu, w którym ograniczenie na całkowitoliczbowość jest ważna}    
\end{figure}
Mamy następujące dane:
\begin{itemize}
    \item $N$ = 4
    \item $i$ = 1
    \item $j$ = 3
    \item $T_{max}$ = 8
    \item $arcs$ - jak na rysunku poniżej
\end{itemize}
Bez ograniczenia na całkowitoliczbowość otrzymuejmy rozwiązanie:
\begin{itemize}
    \item Minimalny koszt: 4.0
    \item Czas przejazdu: 8.0
    \item Ścieżka: ---
\end{itemize}
Do wyboru mieliśmy dwie ścieżki:
\begin{itemize}
    \item Ścieżka 1: $ 1 \rightarrow 2 \rightarrow 3$ o koszcie 6 i czasie 6
    \item Ścieżka 2: $ 1 \rightarrow 4 \rightarrow 3$ o koszcie 2 i czasie 10
\end{itemize}
Solver, chcąc zminimalizować koszt patrzy na tańszą ścieżkę drugą, która nie mieści się w ograniczeniu czasowym, ale 
ponieważ pozwoliliśmy na wartości ułamkowe, solver znajduje matematycznie poprawne, ale logicznie bezsensowne rozwiązanie,
które polega na wysyłaniu części przepływu jedną ścieżką i pozostałej części przepływu drugą ścieżką.
\newpage
Zauważmy również, że po usunięciu ograniczenia na czas przejazdu oraz całkowitoliczbowość dostajemy problem najtańszego przepływu w sieci,
więc otrzymane rozwiązania będą akceptowalne.


\end{document} 