\documentclass{article}
\usepackage[utf8]{inputenc}
\usepackage[T1]{fontenc}
\usepackage{lmodern}
\usepackage[polish]{babel}
\usepackage{amsmath}
\usepackage{tikz}
\usepackage{algorithm}
\usepackage{algpseudocode}
\usepackage{hyperref}
\usepackage{float}
\usepackage{graphicx}
\usepackage{mathtools}
\usepackage{amsmath}
\usepackage{amsfonts}
\usepackage{amsmath}
\usepackage{amsmath}
\usepackage{booktabs}
\usepackage[margin=1in]{geometry}
\usepackage{booktabs} % Dla ładniejszych linii w tabeli
\usepackage{array}




\title{Teoria Grafów}
\author{Wojciech Typer}
\date{}

\begin{document}
\maketitle

\begin{center}
    {\LARGE \textbf{Wprowadzenie do grafów prostych}}
\end{center}

\vspace{2\baselineskip}

\textbf{Literatura:}
\begin{itemize}
    \item R.J. Wilson, \textit{Wprowadzenie do teorii grafów}
    \item D.B. West, \textit{Introduction to Graph Theory}
\end{itemize}

\vspace{1\baselineskip}

\textbf{Definicja:}\\
Grafem prostym $G$ nazywamy parę zbiorów rozłącznych $(V, E)$ takich, że $E \subseteq V^{(2)}$, gdzie $V^{(2)}$ to zbiór wszystkich 2-elementowych 
podzbiorów zbioru $V$. Elementy zbioru $V$ nazywamy \textbf{wierzchołkami}, a elementy zbioru $E$ -- \textbf{krawędziami} grafu $G$. Zbiór $V$ nazywamy zbiorem wierzchołków grafu $G$, a zbiór $E$ -- zbiorem krawędzi grafu $G$.

\vspace{0.5\baselineskip}
\textbf{Oznaczenia:}
\begin{itemize}
    \item $V$ -- zbiór wierzchołków
    \item $E$ -- zbiór krawędzi
\end{itemize}

\noindent
Jeżeli dwie krawędzie mają punkt wspólny, to mówimy, że są to \textbf{krawędzie incydentne}.

\vspace{2\baselineskip}
\textbf{Przykład grafu prostego:}

Załóżmy, że $V = \{ a, b, c, d, e \}$, a $E = \{ \{a, b\}, \{a, c\}, \{b, c\}, \{c, d\}, \{d, e\} \}$.\\
Poniżej znajduje się wizualizacja tego grafu:

\begin{center}
\begin{tikzpicture}[scale=1.2, every node/.style={circle, draw, fill=white, minimum size=8mm, inner sep=0pt}]
    % Wierzchołki
    \node (a) at (0,1.5) {$a$};
    \node (b) at (-1,0) {$b$};
    \node (c) at (1,0) {$c$};
    \node (d) at (2,-1) {$d$};
    \node (e) at (3,0) {$e$};
    % Krawędzie
    \draw (a) -- (b);
    \draw (a) -- (c);
    \draw (b) -- (c);
    \draw (c) -- (d);
    \draw (d) -- (e);
\end{tikzpicture}
\end{center}

\newpage

\begin{center}
\begin{tikzpicture}[scale=1.3, every node/.style={circle, draw, fill=white, minimum size=8mm, inner sep=0pt}]
    % Wierzchołki
    \node (a) at (0,1.5) {$a$};
    \node (b) at (-1,0) {$b$};
    \node (c) at (1,0) {$c$};
    % Krawędzie
    \draw (a) edge[loop above] (a); % pętla przy a
    \draw (b) -- (a);
    \draw (c) -- (a);
    \draw (b) to[bend left=25] (c); % pierwsza krawędź b-c
    \draw (b) to[bend right=25] (c); % druga krawędź b-c (multikrawędź)
\end{tikzpicture}
\end{center}
Powyższy graf jest multigrafem, zawierający multi krawędź między wierzchołkami $b$ i $c$ i pętlę przy wierzchołku $a$. \\
\textbf{Definicja:}\\
Grafem ogólnym G nazywamy trójkę uporządkowaną $(V, E, \phi)$, gdzie $V$ i $E$ są zbiorami rozłącznymi, a $\phi$ jest funkcją przyporządkowującą każdej krawędzi z $E$ jeden lub dwa (niekoniecznie różne) wierzchołki z $V$. Funkcję $\phi$ nazywamy \textbf{funkcją incydencji}. \\
\textbf{Przykład}
$G = (V, E, \phi)$ \\
$V = {1, 2, 3}$ \\
$E = {a, b, c, d}$ \\
$\phi(a) = \{1, 2\}$ \\
$\phi(b) = \{1, 2\}$ \\
$\phi(c) = \{2, 3\}$ \\
$\phi(d) = \{3\}$ \\ %pętla
\begin{center}
\begin{tikzpicture}[scale=1.5, every node/.style={circle, draw, fill=white, minimum size=8mm, inner sep=0pt}]
    % Wierzchołki
    \node (1) at (0,0) {1};
    \node (2) at (2,0) {2};
    \node (3) at (1,1.5) {3};

    % Krawędzie
    \draw[thick] (1) to[bend left=25] node[midway, below] {$a$} (2);   % a: 1-2
    \draw[thick] (1) to[bend right=25] node[midway, below] {$b$} (2);  % b: 1-2 (druga krawędź)
    \draw[thick] (2) -- node[midway, right] {$c$} (3);                 % c: 2-3
    \draw[thick] (3) edge[loop above, looseness=20] node[right] {$d$} (3); % d: pętla na 3
\end{tikzpicture}
\end{center}
\textbf{Definicja:}\\
Niech $G = (V, E, j)$ będzie grafem ogólnym, $v \in V$.\\
\textbf{Stopniem} wierzchołka $v$ nazywamy liczbę:
\[
\deg(v) = 2\, \left| \left\{ e \in E_{1} : v \in j(e) \right\} \right| + \left| \left\{ e \in E_{2} : v \in j(e) \right\} \right|
\]
gdzie:
\begin{itemize}
    \item $E_{1}$ --- zbiór pętli,
    \item $E_{2}$ --- zbiór krawędzi niebędących pętlami.
\end{itemize}
\newpage
\textbf{Lemat Eulera o uściskach dłoni}\\
Niech $G = (V, E, j)$ będzie grafem ogólnym.

Wówczas zachodzi:
\[
\sum_{v \in V} \deg(v)
= \sum_{v \in V} \left( 2 \cdot \sum_{e \in E_1} [v \in j(e)] + \sum_{e \in E_2} [v \in j(e)] \right)
\]
gdzie:
\begin{itemize}
    \item $E_1$ --- zbiór pętli,
    \item $E_2$ --- zbiór krawędzi nie będących pętlami,
\end{itemize}

\noindent
Ponieważ każda pętla jest incydentna tylko z jednym wierzchołkiem, lecz do stopnia liczymy ją podwójnie, oraz każda krawędź nie będąca pętlą jest incydentna z dwoma (różnymi) wierzchołkami, mamy:
\[
\sum_{v \in V} \deg(v)
= 2|E_1| + 2|E_2| = 2(|E_1| + |E_2|) = 2|E|
\]

\noindent
Zatem suma stopni wszystkich wierzchołków w grafie ogólnym równa się dwukrotności liczby krawędzi:
\[
\boxed{
\sum_{v \in V} \deg(v) = 2|E|
}
\]
\textbf{Definicja:}\\
Grafy $G_{1} = (V_{1}, E_{1})$ oraz $G_{2} = (V_{2}, E_{2})$ są \textbf{izomorficzne}, jeśli istnieje bijekcja
\[
\varphi: V_{1} \to V_{2}
\]
taka, że dla każdych $v, w \in V_{1}$ zachodzi:
\[
\{v, w\} \in E_{1} \iff \{\varphi(v), \varphi(w)\} \in E_{2}.
\]
\newline
\textbf{Definicja:}\\
Graf $G = (V, E)$ nazywamy \textbf{dwudzielnym}, jeżeli istnieją rozłączne, niepuste zbiory $A, B \subseteq V$ takie, że:
\begin{itemize}
    \item $A \cap B = \emptyset$, \quad $A \neq \emptyset$, \quad $B \neq \emptyset$,
    \item $A \cup B = V$,
    \item każda krawędź $e = \{v, w\} \in E$ spełnia: $v \in A$ oraz $w \in B$ (lub odwrotnie).
\end{itemize}

\vspace{1em}

\textbf{Definicja:}\\
Niech $G = (V, E)$ będzie grafem prostym. \textbf{Trójkątem} nazywamy trójkę parami różnych wierzchołków $v, w, z \in V$, takich że:
\[
\{v, w\} \in E, \quad \{w, z\} \in E, \quad \{v, z\} \in E
\]

\vspace{1em}
\textbf{Przykład:}\\
Poniżej znajduje się graf będący trójkątem:

\begin{center}
\begin{tikzpicture}[scale=1.2, every node/.style={circle, draw, fill=white, minimum size=8mm, inner sep=0pt}]
    \node (v) at (0,0) {$v$};
    \node (w) at (2,0) {$w$};
    \node (z) at (1,1.73) {$z$};
    \draw (v) -- (w) -- (z) -- (v);
\end{tikzpicture}
\end{center}
\newpage
\textbf{Przykłady grafów:}\\
\begin{itemize}
    \item Graf pełny

    \begin{center}
    \begin{tikzpicture}[scale=1, every node/.style={circle, draw, fill=white, minimum size=7mm, inner sep=0pt}]
        \node (a) at (0,1) {1};
        \node (b) at (1,2) {2};
        \node (c) at (2,1) {3};
        \node (d) at (1,0) {4};
        \foreach \i/\j in {a/b, a/c, a/d, b/c, b/d, c/d}
            \draw (\i) -- (\j);
    \end{tikzpicture}
    \end{center}

    \item Graf liniowy

    \begin{center}
    \begin{tikzpicture}[scale=1, every node/.style={circle, draw, fill=white, minimum size=7mm, inner sep=0pt}]
        \node (a) at (0,0) {1};
        \node (b) at (1,0) {2};
        \node (c) at (2,0) {3};
        \node (d) at (3,0) {4};
        \draw (a) -- (b) -- (c) -- (d);
    \end{tikzpicture}
    \end{center}

    \item Cykl

    \begin{center}
    \begin{tikzpicture}[scale=1, every node/.style={circle, draw, fill=white, minimum size=7mm, inner sep=0pt}]
        \node (a) at (0,1) {1};
        \node (b) at (1,2) {2};
        \node (c) at (2,1) {3};
        \node (d) at (1,0) {4};
        \draw (a) -- (b) -- (c) -- (d) -- (a);
    \end{tikzpicture}
    \end{center}

    \item Graf pełny dwudzielny

    \begin{center}
    \begin{tikzpicture}[scale=1, every node/.style={circle, draw, fill=white, minimum size=7mm, inner sep=0pt}]
        \node (a) at (0,1) {$A_1$};
        \node (b) at (0,-1) {$A_2$};
        \node (c) at (2,0.7) {$B_1$};
        \node (d) at (2,-0.7) {$B_2$};
        \foreach \i in {a,b}
            \foreach \j in {c,d}
                \draw (\i) -- (\j);
    \end{tikzpicture}
    \end{center}

    \item Gwiazda
    \begin{center}
    \begin{tikzpicture}[scale=1.1, every node/.style={circle, draw, fill=white, minimum size=7mm, inner sep=0pt}]
        \node (c) at (0,0) {$v_0$};
        \node (a) at (1.5,1) {$v_1$};
        \node (b) at (1.5,0) {$v_2$};
        \node (d) at (1.5,-1) {$v_3$};
        \draw (c) -- (a);
        \draw (c) -- (b);
        \draw (c) -- (d);
    \end{tikzpicture}
    \end{center}

    \textbf{Graf koło:}

    \begin{center}
    \begin{tikzpicture}[scale=1.2, every node/.style={circle, draw, fill=white, minimum size=7mm, inner sep=0pt}]
        % Wierzchołki na okręgu
        \foreach \i/\name in {1/a,2/b,3/c,4/d,5/e}
            \node (\name) at ({90+72*(\i-1)}:1.5) {$v_{\i}$};
        % Wierzchołek centralny
        \node (o) at (0,0) {$v_0$};
        % Krawędzie cyklu
        \foreach \i/\j in {a/b, b/c, c/d, d/e, e/a}
            \draw (\i) -- (\j);
        % Krawędzie do środka
        \foreach \name in {a,b,c,d,e}
            \draw (o) -- (\name);
    \end{tikzpicture}
    \end{center}
\end{itemize}
\newpage
\textbf{Definicja (dopełnienie grafu):}\\
Dopełnieniem grafu $G = (V, E)$ nazywamy graf $\overline{G} = (V, \overline{E})$, gdzie $\overline{E}$ jest zbiorem wszystkich krawędzi, które nie należą do $E$, tzn.\
\[
\overline{E} = \{\{v, w\} : v, w \in V, v \neq w, \{v, w\} \notin E\}
\]

\vspace{1em}

\textbf{Definicja (suma dwóch grafów):}\\
Sumą grafów $G_1 = (V_1, E_1)$ oraz $G_2 = (V_2, E_2)$ (dla $V_1 \cap V_2 = \emptyset$) nazywamy graf $G = (V_1 \cup V_2, E_1 \cup E_2)$.
\newline
\textbf{Lemat 1}\\
Niech $G = (V, E)$ będzie grafem prostym bez trójkątów. Wtedy dla każdej krawędzi $\{v, w\} \in E$ zachodzi:
\begin{center}
    \[
    \deg(v) + \deg(w) \leq n = |V|
    \]
\end{center}

\vspace{1em}

\textbf{Lemat 2}\\
Niech $G = (V, E)$. Wówczas:
\begin{center}
    \[
    \sum_{\{v, w\} \in E} \left( \deg(v) + \deg(w) \right) = \sum_{v \in V} \left(\deg(v)\right)^2
    \]
\end{center}

\textbf{Twierdzenie (Mantela):}\\
Niech $G = (V, E)$ będzie grafem prostym o $n \geq 3$ wierzchołkach, w którym nie ma trójkąta (czyli graf nie zawiera cyklu długości 3). Wówczas:
\[
|E| \leq \left\lfloor \frac{n^2}{4} \right\rfloor.
\]
Osiągnięcie tej liczby krawędzi jest możliwe tylko wtedy, gdy $G$ jest grafem pełnym dwudzielnym z częściami o rozmiarach $\left\lfloor \frac{n}{2} \right\rfloor$ i $\left\lceil \frac{n}{2} \right\rceil$.

\vspace{1em}

\textbf{Dowód:}\\
Załóżmy, że $G = (V, E)$ jest grafem prostym bez trójkąta, $|V| = n$. Niech $A$ i $B$ będą dwoma rozłącznymi podzbiorami $V$ takimi, że $A \cup B = V$ i $|A| = \left\lfloor \frac{n}{2} \right\rfloor$, $|B| = \left\lceil \frac{n}{2} \right\rceil$. 

Każda krawędź w grafie dwudzielnym $K_{|A|, |B|}$ łączy wierzchołek z $A$ z wierzchołkiem z $B$, więc liczba krawędzi wynosi $|A| \cdot |B| = \left\lfloor \frac{n}{2} \right\rfloor \cdot \left\lceil \frac{n}{2} \right\rceil = \left\lfloor \frac{n^2}{4} \right\rfloor$. 

Pokażemy, że żaden graf prosty bez trójkąta nie może mieć więcej krawędzi. Bez straty ogólności, dla dowolnej krawędzi $\{v, w\}$ wszystkie sąsiady $v$ i $w$ są różne, bo inaczej powstałby trójkąt. Zatem suma stopni wszystkich wierzchołków jest ograniczona, a dokładniej liczba krawędzi jest maksymalna wtedy, gdy $G$ jest kompletnym grafem dwudzielnym, czyli $|E| \leq \left\lfloor \frac{n^2}{4} \right\rfloor$. 

\vspace{0.5em}
\textit{Dowód} Niech $v$ będzie wierzchołkiem o największym stopniu $d$. Jego sąsiedzi nie mogą być ze sobą połączeni, więc mogą mieć krawędzie tylko do pozostałych $n-d-1$ wierzchołków. Zliczając krawędzie i maksymalizując wyrażenie, otrzymujemy ograniczenie $|E| \leq \frac{n^2}{4}$. 

\textbf{Wniosek:} Najwięcej krawędzi w grafie prostym bez trójkąta ma graf pełny dwudzielny z częściami możliwie równymi. \\
% Twierdzenie Turana
\newpage
\textbf{Graf Eulerowski:} \\
Niech $G = (V, E, \gamma)$ będzie grafem ogólnym.
\begin{itemize}
    \item Trasa to ciąg $v_{0} e_{1} v_{1} e_{2} v_{2} \dots$ taki, że $v_{0}, v_{1}, \dots \in V$ i $e_{1}, e_{2}, \dots \in E$
    \item Ścieżka to trasa, która nie powtarza krawędzi
    \item Ścieżka zamknięta to ścieżka, w której $v_{0} = v_{k}$ (zaczyna się i kończy w tym samym wierzchołku)
    \item Droga to ścieżka, która nie powtarza wierzchołków.
    \item Cykl to ścieżka, w której wierzchołki się nie powtarzają, poza $v_{0} = v_{k}$
\end{itemize}
\textbf{Definicja:} Niech $G = (V, E, \gamma), v,w \in V$. Odległością $v$ od $w$ nazywamy $d(v,w)$ - długość najkrótszej drogi z $v$ do $w$. 
Jeżeli taka droga nie istnieje, to $d(v, w) = \infty$. \\
\textbf{Definicja} Niech $G = (V, E, \gamma)$ G jest spójny, jeżeli: $\forall v, w \in V d(v, w) < \infty$ \\
\textbf{Fakt: } Na zbiorze V można wprowadzić relację równoważności: \\
\begin{center}
    $\forall v, w \in V v~w \equiv d(v, w) < \infty$
\end{center}
Klasy abstrakcji relacji ~ definiujemy tzw. spójne składowe (komponenty) grafu G \\
\textbf{Definicja: } Niech $G = (V, E, \gamma)$. G nazywamy eulerowskim, jeżeli w G istnieje ścieżka zamknięta, zawierająca każdą krawędź z E.
%Tu daj przykład grafu eulerowskiego 6 wierzchołkowego
\begin{center}
\begin{tikzpicture}[
    v/.style={circle, draw, fill=black!10, minimum size=22pt, inner sep=0pt},
    every edge/.style={draw, thick}
]
    % Definicja wierzchołków
    \node[v] (v1) at (90:2) {$v_1$};
    \node[v] (v2) at (30:2) {$v_2$};
    \node[v] (v3) at (330:2) {$v_3$};
    \node[v] (v4) at (270:2) {$v_4$};
    \node[v] (v5) at (210:2) {$v_5$};
    \node[v] (v6) at (150:2) {$v_6$};
    
    % Definicja krawędzi
    \path
        % Zewnętrzny cykl (stopień 2 dla każdego wierzchołka)
        (v1) edge (v2)
        (v2) edge (v3)
        (v3) edge (v4)
        (v4) edge (v5)
        (v5) edge (v6)
        (v6) edge (v1)
        % Trójkąt wewnętrzny (dodaje stopień 2 do v1, v3, v5)
        (v1) edge (v3)
        (v3) edge (v5)
        (v5) edge (v1);
\end{tikzpicture}
\end{center}
\textbf{Definicja: } Niech $G = (V, E, \gamma)$. G nazywamy pół eulerowskim, jeśli G nie jest eulerowski oraz w G istnieje ścieżka zawierająca każda krawędź z E
\begin{center}
\begin{tikzpicture}[
    v/.style={circle, draw, fill=black!10, minimum size=22pt, inner sep=0pt},
    every edge/.style={draw, thick}
]
    % Definicja wierzchołków w układzie kołowym
    \node[v] (v1) at (90:2) {$v_1$};
    \node[v] (v2) at (30:2) {$v_2$};
    \node[v] (v3) at (330:2) {$v_3$};
    \node[v] (v4) at (270:2) {$v_4$};
    \node[v] (v5) at (210:2) {$v_5$};
    \node[v] (v6) at (150:2) {$v_6$};
    
    % Definicja krawędzi
    \path
        % Zewnętrzny cykl (stopień 2 dla każdego wierzchołka)
        (v1) edge (v2)
        (v2) edge (v3)
        (v3) edge (v4)
        (v4) edge (v5)
        (v5) edge (v6)
        (v6) edge (v1)
        % Krawędź wewnętrzna łącząca v1 i v4
        % Zwiększa stopień v1 i v4 o 1, tworząc dwa wierzchołki nieparzyste
        (v1) edge (v4);
\end{tikzpicture}
\end{center}
\newpage
\textbf{Lemat: }Niech $G = (V, E, \gamma)$. Jeżeli $\forall v \in V$, deg(v) $\geq$ 2, to w G występuje cykl. \\
\textbf{Tw. Eulera, 1736} Niech $G = (V, E, \gamma)$ G jest eulerowski $\equiv$ G jest spójny i $\forall v \in V$ 2 | deg(v) \\
\textbf{Definicja: }Niech $G = (V, E, \gamma)$, c(G) oznacza liczbe komponent grafu G. Krawędź $e \in E$ nazywamy mostem, jeżeli c(G / e) (graf G po usunięciu krawędzi e)
> c(G) 
%Algorytm Flammyego
\newpage
\subsection*{Grafy Hamiltonowskie}

Graf $G = (V, E)$ nazywamy \emph{hamiltonowskim}, jeśli w G istnieje cykl, który zawiera każdy wierzchołek z V (dokładnie jeden raz).

\begin{quote}
\textbf{Definicja:} Graf $G = (V, E)$ nazywamy \emph{półhamiltonowskim}, jeżeli G nie jest hamiltonowski i w G istnieje droga zawierająca każdy wierzchołek z V (dokładnie jeden raz). Ta droga nazywana jest drogą lub ścieżką Hamiltona.
\end{quote}

\begin{center}
  % Graf Hamiltonowski (cykl)
  \begin{tikzpicture}[
      v/.style={circle, draw, fill=black!10, minimum size=22pt, inner sep=0pt},
      every edge/.style={draw, thick}
  ]
      % Wierzchołki
      \node[v] (h1) at (90:1.5)  {$v_1$};
      \node[v] (h2) at (0:1.5)   {$v_2$};
      \node[v] (h3) at (270:1.5) {$v_3$};
      \node[v] (h4) at (180:1.5) {$v_4$};
      % Krawędzie
      \path
        (h1) edge (h2)
        (h2) edge (h3)
        (h3) edge (h4)
        (h4) edge (h1);
      % Podpis
      %\node[font=\bfseries, below=0.5cm of h3] {Graf hamiltonowski};
  \end{tikzpicture}
  \qquad\qquad % Odstęp poziomy
  % Graf półhamiltonowski (ścieżka)
  \begin{tikzpicture}[
      v/.style={circle, draw, fill=black!10, minimum size=22pt, inner sep=0pt},
      every edge/.style={draw, thick}
  ]
      % Wierzchołki
      \node[v] (p1) at (0,0) {$v_1$};
      \node[v] (p2) at (2,0) {$v_2$};
      \node[v] (p3) at (4,0) {$v_3$};
      \node[v] (p4) at (6,0) {$v_4$};
      % Krawędzie
      \path
        (p1) edge (p2)
        (p2) edge (p3)
        (p3) edge (p4);
      % Podpis (poprawione pozycjonowanie)
      \node[font=\bfseries] at (3, -1.2) {Graf półhamiltonowski};
  \end{tikzpicture}
\end{center}

\noindent\textbf{Uwaga:} Pętle i multikrawędzie nie mają wpływu na rozważania nad hamiltonowskością grafu, zatem ograniczamy się do grafów prostych.

\vspace{1em} % Dodatkowy odstęp przed tabelą

% Tabela bez pakietu booktabs
\begin{table}[h!]
\centering
\caption{Porównanie cyklu Eulera i cyklu Hamiltona}
\begin{tabular}{|p{0.45\textwidth}|p{0.45\textwidth}|}
\hline
\textbf{Cykl Eulera} & \textbf{Cykl Hamiltona} \\
\hline
Warunkiem koniecznym i wystarczającym istnienia w grafie spójnym jest to, aby każdy wierzchołek miał stopień parzysty. & Odwiedza każdy wierzchołek grafu dokładnie jeden raz. \\[4mm]
Każda krawędź musi być użyta dokładnie jeden raz. & Może pomijać niektóre krawędzie, aby uniknąć ponownego odwiedzania wierzchołków. \\[4mm]
Istnieją algorytmy o złożoności wielomianowej znajdujące cykl (np. algorytm Hierholzera O($|E|$)). & Problem decyzyjny jest NP-zupełny. Nie jest znany algorytm o złożoności wielomianowej, który by orzekał, czy dany graf jest hamiltonowski. \\
\hline
\end{tabular}
\end{table}
\begin{quote}
\textbf{Twierdzenie Diraca (1952):} Niech $G = (V, E)$ będzie grafem prostym o $|V| \geq 3$ oraz $\forall v \in V: \deg(v) \geq \frac{|V|}{2}$. Wówczas G jest hamiltonowski.
\end{quote}

\begin{quote}
\textbf{Twierdzenie Orego (1960):} Niech $G = (V, E)$ będzie grafem prostym o $|V| \geq 3$ oraz dla każdej pary niepołączonych krawędzią wierzchołków $\{v, w\}$ zachodzi $\deg(v) + \deg(w) \geq |V|$. Wówczas G jest hamiltonowski.
\end{quote}
\newpage
%ćwiczenie: zaproponuj algorytm o złożoności wielomianowej, który znajduje cykl hamiltona w grafie spełniającym założenia twierdzenia orego
\section*{Ćwiczenia - Lista 1}

\subsection*{Zadanie 1/1}

Wiemy, że ilość wszystkich par wierzchołków w grafie prostym $G = (V, E)$ o $n$ wierzchołkach wynosi $\binom{n}{2}$.
Każdą parę możemy połączyć krawędzią lub nie.
Zatem ilość wszystkich grafów prostych na $n$ wierzchołkach wynosi:
\[
2^{\binom{n}{2}}
\]

\textbf{Pytanie:} Ile z nich ma dokładnie $m$ krawędzi? \\
Jest to równoważne z wyborem $m$ krawędzi spośród wszystkich $\binom{n}{2}$ możliwych, zatem:
\[
\binom{\binom{n}{2}}{m}
\]

\subsection*{zadanie 1/2}
\textbf{Pytanie:} Czy istnieje graf prosty o co najmniej dwóch wierzchołkach, w którym wszystkie wierzchołki mają różne stopnie?

Taki graf $n$-wierzchołkowy musiałby mieć wierzchołki o stopniach: $0, 1, 2, \dots, n-1$.
Wierzchołek o stopniu $n-1$ jest połączony z wszystkimi innymi wierzchołkami, co oznacza, że nie może istnieć wierzchołek o stopniu 0 (izolowany).
Zatem nie istnieje graf prosty o co najmniej dwóch wierzchołkach, w którym wszystkie wierzchołki mają różne stopnie.

\subsection*{zadanie 1/3}
\textbf{Pytanie:} Czy suma stopni wszystkich wierzchołków w grafie prostym może być nieparzysta?

Nie, ponieważ zgodnie z lematem o uściskach dłoni (handshaking lemma), suma stopni wszystkich wierzchołków w grafie jest równa podwojonej liczbie krawędzi ($ \sum_{v \in V} \deg(v) = 2|E| $), a więc jest zawsze liczbą parzystą.

\newpage
\section*{Ćwiczenia - Lista 2}
\subsection*{zadanie 2/1}
Wiemy, że graf $G = (V, E)$ jest grafem prostym bez trójkątów - nie pojawiają się w nim podgrafy o trzech wierzchołkach, gdzie każdy wierzchołek jest połączony z pozostałymi dwoma.

Oznaczmy: $N(x)$ - zbiór sąsiadów wierzchołka $x$ w grafie $G$. Z własności grafu bez trójkątów wynika, że $N(v) \cap N(w) = \emptyset$ dla każdej krawędzi $\{v, w\} \in E$.

Ponieważ zbiory $N(v)$ i $N(w)$ są rozłączne, suma mocy ich zbiorów jest równa mocy ich unii: $|N(v)| + |N(w)| = |N(v) \cup N(w)|$. Zbiór $N(v) \cup N(w)$ jest podzbiorem $V$, więc $|N(v) \cup N(w)| \leq |V|$.
Zatem $\deg(v) + \deg(w) \leq |V|$. \\

Dowód własności z trójkątem:
Niech ${v, w} \in E$. Załóżmy, że $N(v) \cap N(w) \neq \emptyset$. Wowczas wynika z tego: \\
$\exists u \in V: u \in N(v) \land u \in N(w)$ \\
Zatem $\{u, v\} \in E \land \{u, w\} \in E$, co oznacza, że wierzchołki $u, v, w$ tworzą trójkąt, co jest sprzeczne z założeniem.

\subsection*{zadanie 2/2}
Niech $G = (V, E)$ będzie grafem prostym. Musimy uzasadnić poniższe równanie:
$\sum_{\{v, w\} \in E} (\deg(v) + \deg(w)) = \sum_{v \in V} (\deg(v))^2$. \\
Po lewej stronie sumujemy dla każdej krawędzi $\{v, w\}$ sumę stopni jej końców. Oznacza to, że każdy wierzchołek $v$ jest liczony dokładnie $\deg(v)$ razy (raz dla każdej krawędzi incydentnej z $v$). Zatem lewa strona równania to:
\[\sum_{v \in V} \deg(v) \cdot \deg(v) = \sum_{v \in V} (\deg(v))^2\]
co jest dokładnie prawą stroną równania.

\newpage

\subsection*{zadanie 2/3}
\begin{figure}[h!]
    \centering
    % Skalujemy obrazek do 80% szerokości tekstu
    \includegraphics[width=0.8\textwidth]{/home/wojteq18/latex_photo/1.png}
    \caption{Hiperkostka w kolejnych wymiarach.}
    \label{fig:obrazek1}
\end{figure}

W $Q_{k}$
\begin{itemize}
    \item ilość wierzchołków: $2^k$
    \item stopnie wierzchołków: $k$
    \item ilość krawędzi: Z lematu o uściskach dłoni: $|E| = \frac{2^k \cdot k}{2} = k \cdot 2^{k-1}$, dla $ k \geq 1$
\end{itemize}

\textbf{Średnica hiperkostki:} \\
Średnica hiperkostki $Q_k$ wynosi $k$. Wynika to z faktu, że hiperkostka formalnie definiowana jest jako graf, w którym wierzchołkami są wszystykie ciągi binarne dlugości k.
Średnica to maksymalna odległość między dwoma wierzchołkami, a odległość między dwoma wierzchołkami w hiperkostce to liczba pozycji, na których ich reprezentacje binarne różnią się (odległość Hamminga). 
Największa możliwa odległość występuje między wierzchołkami reprezentowanymi przez ciągi $000...0$ i $111...1$, które różnią się na wszystkich $k$ pozycjach. Zatem średnica hiperkostki $Q_k$ wynosi $k$. \\

\textbf{$Q_{k}$ jako graf dwudzielny:} \\
W hiperkostka $Q_k$ każdy wierzchołek można zdefiniować jako ciąg binarny długości $k$. Każdy wierzchołek łączy się z innymi, 
wtedy, gdy ich reprezentacje różnią się dokładnie na jednej pozycji. Możemy podzielić wierzchołki na dwa zbiory:
\begin{itemize}
    \item $A$ - wierzchołki z parzystą liczbą jedynek w reprezentacji binarnej,
    \item $B$ - wierzchołki z nieparzystą liczbą jedynek w reprezentacji binarnej.
\end{itemize}   
W ten sposób widać, że graf $Q_k$ jest dwudzielny.

\newpage
\subsection*{zadanie 2/4}
$K_{2, 2}$ to graf dwudzielny, w którym wierzchołki są podzielone na dwa zbiory, oba zawierające po 2 wierzchołki.

\subsection*{zadanie 2/5}
Niech $G = (V, E)$ będzie grafem dwudzielnym z trójkątem. Oznacza to, że: \\
$\exists (A, B \subseteq V) (A \cap B = \emptyset \land A \cup B = V \land \forall {w, v} \in E (w \in A \land v \in B))$ \\
Weźmy jeden z takich trójkątów i pokolorujmy jego wierzchołki na dwa kolory, tak aby żadne dwa sąsiednie wierzchołki nie miały tego samego koloru. Ponieważ trójkąt ma trzy wierzchołki, a my mamy tylko dwa kolory, to zgodnie z zasadą szufladkową, co najmniej dwa wierzchołki muszą być tego samego koloru. Jednak te dwa wierzchołki są połączone krawędzią (bo są częścią trójkąta), co jest sprzeczne z założeniem, że żadne dwa sąsiednie wierzchołki nie mogą mieć tego samego koloru. \\

\end{document}
