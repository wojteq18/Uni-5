\documentclass{article}
\usepackage[utf8]{inputenc}
\usepackage[T1]{fontenc}
\usepackage{lmodern}
\usepackage[polish]{babel}
\usepackage{amsmath}
\usepackage{tikz}
\usepackage{algorithm}
\usepackage{algpseudocode}
\usepackage{hyperref}
\usepackage{float}
\usepackage{graphicx}
\usepackage{mathtools}
\usepackage{amsmath}
\usepackage{amsfonts}
\usepackage{amsmath}
\usepackage{amsmath}
\usepackage{booktabs}
\usepackage[margin=1in]{geometry}




\title{Teoria Grafów}
\author{Wojciech Typer}
\date{}

\begin{document}
\maketitle

\begin{center}
    {\LARGE \textbf{Wprowadzenie do grafów prostych}}
\end{center}

\vspace{2\baselineskip}

\textbf{Literatura:}
\begin{itemize}
    \item R.J. Wilson, \textit{Wprowadzenie do teorii grafów}
    \item D.B. West, \textit{Introduction to Graph Theory}
\end{itemize}

\vspace{1\baselineskip}

\textbf{Definicja:}\\
Grafem prostym $G$ nazywamy parę zbiorów rozłącznych $(V, E)$ takich, że $E \subseteq V^{(2)}$, gdzie $V^{(2)}$ to zbiór wszystkich 2-elementowych 
podzbiorów zbioru $V$. Elementy zbioru $V$ nazywamy \textbf{wierzchołkami}, a elementy zbioru $E$ -- \textbf{krawędziami} grafu $G$. Zbiór $V$ nazywamy zbiorem wierzchołków grafu $G$, a zbiór $E$ -- zbiorem krawędzi grafu $G$.

\vspace{0.5\baselineskip}
\textbf{Oznaczenia:}
\begin{itemize}
    \item $V$ -- zbiór wierzchołków
    \item $E$ -- zbiór krawędzi
\end{itemize}

\noindent
Jeżeli dwie krawędzie mają punkt wspólny, to mówimy, że są to \textbf{krawędzie incydentne}.

\vspace{2\baselineskip}
\textbf{Przykład grafu prostego:}

Załóżmy, że $V = \{ a, b, c, d, e \}$, a $E = \{ \{a, b\}, \{a, c\}, \{b, c\}, \{c, d\}, \{d, e\} \}$.\\
Poniżej znajduje się wizualizacja tego grafu:

\begin{center}
\begin{tikzpicture}[scale=1.2, every node/.style={circle, draw, fill=white, minimum size=8mm, inner sep=0pt}]
    % Wierzchołki
    \node (a) at (0,1.5) {$a$};
    \node (b) at (-1,0) {$b$};
    \node (c) at (1,0) {$c$};
    \node (d) at (2,-1) {$d$};
    \node (e) at (3,0) {$e$};
    % Krawędzie
    \draw (a) -- (b);
    \draw (a) -- (c);
    \draw (b) -- (c);
    \draw (c) -- (d);
    \draw (d) -- (e);
\end{tikzpicture}
\end{center}

\newpage

\begin{center}
\begin{tikzpicture}[scale=1.3, every node/.style={circle, draw, fill=white, minimum size=8mm, inner sep=0pt}]
    % Wierzchołki
    \node (a) at (0,1.5) {$a$};
    \node (b) at (-1,0) {$b$};
    \node (c) at (1,0) {$c$};
    % Krawędzie
    \draw (a) edge[loop above] (a); % pętla przy a
    \draw (b) -- (a);
    \draw (c) -- (a);
    \draw (b) to[bend left=25] (c); % pierwsza krawędź b-c
    \draw (b) to[bend right=25] (c); % druga krawędź b-c (multikrawędź)
\end{tikzpicture}
\end{center}
Powyższy graf jest multigrafem, zawierający multi krawędź między wierzchołkami $b$ i $c$ i pętlę przy wierzchołku $a$. \\
\textbf{Definicja:}\\
Grafem ogólnym G nazywamy trójkę uporządkowaną $(V, E, \phi)$, gdzie $V$ i $E$ są zbiorami rozłącznymi, a $\phi$ jest funkcją przyporządkowującą każdej krawędzi z $E$ jeden lub dwa (niekoniecznie różne) wierzchołki z $V$. Funkcję $\phi$ nazywamy \textbf{funkcją incydencji}. \\
\textbf{Przykład}
$G = (V, E, \phi)$ \\
$V = {1, 2, 3}$ \\
$E = {a, b, c, d}$ \\
$\phi(a) = \{1, 2\}$ \\
$\phi(b) = \{1, 2\}$ \\
$\phi(c) = \{2, 3\}$ \\
$\phi(d) = \{3\}$ \\ %pętla
\begin{center}
\begin{tikzpicture}[scale=1.5, every node/.style={circle, draw, fill=white, minimum size=8mm, inner sep=0pt}]
    % Wierzchołki
    \node (1) at (0,0) {1};
    \node (2) at (2,0) {2};
    \node (3) at (1,1.5) {3};

    % Krawędzie
    \draw[thick] (1) to[bend left=25] node[midway, below] {$a$} (2);   % a: 1-2
    \draw[thick] (1) to[bend right=25] node[midway, below] {$b$} (2);  % b: 1-2 (druga krawędź)
    \draw[thick] (2) -- node[midway, right] {$c$} (3);                 % c: 2-3
    \draw[thick] (3) edge[loop above, looseness=20] node[right] {$d$} (3); % d: pętla na 3
\end{tikzpicture}
\end{center}
\textbf{Definicja:}\\
Niech $G = (V, E, j)$ będzie grafem ogólnym, $v \in V$.\\
\textbf{Stopniem} wierzchołka $v$ nazywamy liczbę:
\[
\deg(v) = 2\, \left| \left\{ e \in E_{1} : v \in j(e) \right\} \right| + \left| \left\{ e \in E_{2} : v \in j(e) \right\} \right|
\]
gdzie:
\begin{itemize}
    \item $E_{1}$ --- zbiór pętli,
    \item $E_{2}$ --- zbiór krawędzi niebędących pętlami.
\end{itemize}
\newpage
\textbf{Lemat Eulera o uściskach dłoni}\\
Niech $G = (V, E, j)$ będzie grafem ogólnym.

Wówczas zachodzi:
\[
\sum_{v \in V} \deg(v)
= \sum_{v \in V} \left( 2 \cdot \sum_{e \in E_1} [v \in j(e)] + \sum_{e \in E_2} [v \in j(e)] \right)
\]
gdzie:
\begin{itemize}
    \item $E_1$ --- zbiór pętli,
    \item $E_2$ --- zbiór krawędzi nie będących pętlami,
\end{itemize}

\noindent
Ponieważ każda pętla jest incydentna tylko z jednym wierzchołkiem, lecz do stopnia liczymy ją podwójnie, oraz każda krawędź nie będąca pętlą jest incydentna z dwoma (różnymi) wierzchołkami, mamy:
\[
\sum_{v \in V} \deg(v)
= 2|E_1| + 2|E_2| = 2(|E_1| + |E_2|) = 2|E|
\]

\noindent
Zatem suma stopni wszystkich wierzchołków w grafie ogólnym równa się dwukrotności liczby krawędzi:
\[
\boxed{
\sum_{v \in V} \deg(v) = 2|E|
}
\]
\textbf{Definicja:}\\
Grafy $G_{1} = (V_{1}, E_{1})$ oraz $G_{2} = (V_{2}, E_{2})$ są \textbf{izomorficzne}, jeśli istnieje bijekcja
\[
\varphi: V_{1} \to V_{2}
\]
taka, że dla każdych $v, w \in V_{1}$ zachodzi:
\[
\{v, w\} \in E_{1} \iff \{\varphi(v), \varphi(w)\} \in E_{2}.
\]
\newline
\textbf{Definicja:}\\
Graf $G = (V, E)$ nazywamy \textbf{dwudzielnym}, jeżeli istnieją rozłączne, niepuste zbiory $A, B \subseteq V$ takie, że:
\begin{itemize}
    \item $A \cap B = \emptyset$, \quad $A \neq \emptyset$, \quad $B \neq \emptyset$,
    \item $A \cup B = V$,
    \item każda krawędź $e = \{v, w\} \in E$ spełnia: $v \in A$ oraz $w \in B$ (lub odwrotnie).
\end{itemize}

\vspace{1em}

\textbf{Definicja:}\\
Niech $G = (V, E)$ będzie grafem prostym. \textbf{Trójkątem} nazywamy trójkę parami różnych wierzchołków $v, w, z \in V$, takich że:
\[
\{v, w\} \in E, \quad \{w, z\} \in E, \quad \{v, z\} \in E
\]

\vspace{1em}
\textbf{Przykład:}\\
Poniżej znajduje się graf będący trójkątem:

\begin{center}
\begin{tikzpicture}[scale=1.2, every node/.style={circle, draw, fill=white, minimum size=8mm, inner sep=0pt}]
    \node (v) at (0,0) {$v$};
    \node (w) at (2,0) {$w$};
    \node (z) at (1,1.73) {$z$};
    \draw (v) -- (w) -- (z) -- (v);
\end{tikzpicture}
\end{center}
\newpage
\textbf{Przykłady grafów:}\\
\begin{itemize}
    \item Graf pełny

    \begin{center}
    \begin{tikzpicture}[scale=1, every node/.style={circle, draw, fill=white, minimum size=7mm, inner sep=0pt}]
        \node (a) at (0,1) {1};
        \node (b) at (1,2) {2};
        \node (c) at (2,1) {3};
        \node (d) at (1,0) {4};
        \foreach \i/\j in {a/b, a/c, a/d, b/c, b/d, c/d}
            \draw (\i) -- (\j);
    \end{tikzpicture}
    \end{center}

    \item Graf liniowy

    \begin{center}
    \begin{tikzpicture}[scale=1, every node/.style={circle, draw, fill=white, minimum size=7mm, inner sep=0pt}]
        \node (a) at (0,0) {1};
        \node (b) at (1,0) {2};
        \node (c) at (2,0) {3};
        \node (d) at (3,0) {4};
        \draw (a) -- (b) -- (c) -- (d);
    \end{tikzpicture}
    \end{center}

    \item Cykl

    \begin{center}
    \begin{tikzpicture}[scale=1, every node/.style={circle, draw, fill=white, minimum size=7mm, inner sep=0pt}]
        \node (a) at (0,1) {1};
        \node (b) at (1,2) {2};
        \node (c) at (2,1) {3};
        \node (d) at (1,0) {4};
        \draw (a) -- (b) -- (c) -- (d) -- (a);
    \end{tikzpicture}
    \end{center}

    \item Graf pełny dwudzielny

    \begin{center}
    \begin{tikzpicture}[scale=1, every node/.style={circle, draw, fill=white, minimum size=7mm, inner sep=0pt}]
        \node (a) at (0,1) {$A_1$};
        \node (b) at (0,-1) {$A_2$};
        \node (c) at (2,0.7) {$B_1$};
        \node (d) at (2,-0.7) {$B_2$};
        \foreach \i in {a,b}
            \foreach \j in {c,d}
                \draw (\i) -- (\j);
    \end{tikzpicture}
    \end{center}

    \item Gwiazda
    \begin{center}
    \begin{tikzpicture}[scale=1.1, every node/.style={circle, draw, fill=white, minimum size=7mm, inner sep=0pt}]
        \node (c) at (0,0) {$v_0$};
        \node (a) at (1.5,1) {$v_1$};
        \node (b) at (1.5,0) {$v_2$};
        \node (d) at (1.5,-1) {$v_3$};
        \draw (c) -- (a);
        \draw (c) -- (b);
        \draw (c) -- (d);
    \end{tikzpicture}
    \end{center}

    \textbf{Graf koło:}

    \begin{center}
    \begin{tikzpicture}[scale=1.2, every node/.style={circle, draw, fill=white, minimum size=7mm, inner sep=0pt}]
        % Wierzchołki na okręgu
        \foreach \i/\name in {1/a,2/b,3/c,4/d,5/e}
            \node (\name) at ({90+72*(\i-1)}:1.5) {$v_{\i}$};
        % Wierzchołek centralny
        \node (o) at (0,0) {$v_0$};
        % Krawędzie cyklu
        \foreach \i/\j in {a/b, b/c, c/d, d/e, e/a}
            \draw (\i) -- (\j);
        % Krawędzie do środka
        \foreach \name in {a,b,c,d,e}
            \draw (o) -- (\name);
    \end{tikzpicture}
    \end{center}
\end{itemize}
\newpage
\textbf{Definicja (dopełnienie grafu):}\\
Dopełnieniem grafu $G = (V, E)$ nazywamy graf $\overline{G} = (V, \overline{E})$, gdzie $\overline{E}$ jest zbiorem wszystkich krawędzi, które nie należą do $E$, tzn.\
\[
\overline{E} = \{\{v, w\} : v, w \in V, v \neq w, \{v, w\} \notin E\}
\]

\vspace{1em}

\textbf{Definicja (suma dwóch grafów):}\\
Sumą grafów $G_1 = (V_1, E_1)$ oraz $G_2 = (V_2, E_2)$ (dla $V_1 \cap V_2 = \emptyset$) nazywamy graf $G = (V_1 \cup V_2, E_1 \cup E_2)$.
\newline
\textbf{Lemat 1}\\
Niech $G = (V, E)$ będzie grafem prostym bez trójkątów. Wtedy dla każdej krawędzi $\{v, w\} \in E$ zachodzi:
\begin{center}
    \[
    \deg(v) + \deg(w) \leq n = |V|
    \]
\end{center}

\vspace{1em}

\textbf{Lemat 2}\\
Niech $G = (V, E)$. Wówczas:
\begin{center}
    \[
    \sum_{\{v, w\} \in E} \left( \deg(v) + \deg(w) \right) = \sum_{v \in V} \left(\deg(v)\right)^2
    \]
\end{center}

\textbf{Twierdzenie (Mantela):}\\
Niech $G = (V, E)$ będzie grafem prostym o $n \geq 3$ wierzchołkach, w którym nie ma trójkąta (czyli graf nie zawiera cyklu długości 3). Wówczas:
\[
|E| \leq \left\lfloor \frac{n^2}{4} \right\rfloor.
\]
Osiągnięcie tej liczby krawędzi jest możliwe tylko wtedy, gdy $G$ jest grafem pełnym dwudzielnym z częściami o rozmiarach $\left\lfloor \frac{n}{2} \right\rfloor$ i $\left\lceil \frac{n}{2} \right\rceil$.

\vspace{1em}

\textbf{Dowód:}\\
Załóżmy, że $G = (V, E)$ jest grafem prostym bez trójkąta, $|V| = n$. Niech $A$ i $B$ będą dwoma rozłącznymi podzbiorami $V$ takimi, że $A \cup B = V$ i $|A| = \left\lfloor \frac{n}{2} \right\rfloor$, $|B| = \left\lceil \frac{n}{2} \right\rceil$. 

Każda krawędź w grafie dwudzielnym $K_{|A|, |B|}$ łączy wierzchołek z $A$ z wierzchołkiem z $B$, więc liczba krawędzi wynosi $|A| \cdot |B| = \left\lfloor \frac{n}{2} \right\rfloor \cdot \left\lceil \frac{n}{2} \right\rceil = \left\lfloor \frac{n^2}{4} \right\rfloor$. 

Pokażemy, że żaden graf prosty bez trójkąta nie może mieć więcej krawędzi. Bez straty ogólności, dla dowolnej krawędzi $\{v, w\}$ wszystkie sąsiady $v$ i $w$ są różne, bo inaczej powstałby trójkąt. Zatem suma stopni wszystkich wierzchołków jest ograniczona, a dokładniej liczba krawędzi jest maksymalna wtedy, gdy $G$ jest kompletnym grafem dwudzielnym, czyli $|E| \leq \left\lfloor \frac{n^2}{4} \right\rfloor$. 

\vspace{0.5em}
\textit{Szkic dowodu alternatywnego:} Niech $v$ będzie wierzchołkiem o największym stopniu $d$. Jego sąsiedzi nie mogą być ze sobą połączeni, więc mogą mieć krawędzie tylko do pozostałych $n-d-1$ wierzchołków. Zliczając krawędzie i maksymalizując wyrażenie, otrzymujemy ograniczenie $|E| \leq \frac{n^2}{4}$. 

\textbf{Wniosek:} Najwięcej krawędzi w grafie prostym bez trójkąta ma graf pełny dwudzielny z częściami możliwie równymi.
% Twierdzenie Turana
\end{document}
